\documentclass{bmd2019a}

\usepackage{siunitx}
\usepackage{amsmath}
\usepackage{booktabs}

\begin{document}

\begin{flushleft}
{\fontsize{16pt}{20pt}\selectfont%
  Expanded Optimization for Discovering\\}
{\fontsize{16pt}{20pt}\selectfont%
  Optimal Lateral Handling Bicycles}
\end{flushleft}

%%%%%%%%%%%%%%%% authors %%%%%%%%%%%%%%%
\begin{flushleft}
  {\fontsize{12}{14}{Jason K. Moore and Mont Hubbard}\\}
  \textit{Mechanical and Aerospace Engineering\\
          University of California, Davis\\
          One Shields Avenue, Davis, CA, USA 95817\\
          e-mail: jkm@ucdavis.edu, mhubbard@ucdavis.edu}
\end{flushleft}

% - Abstract
% - Introduction
% - Parameterization
% - Bounds and Constraints
% - Free Parameters
% - Optimization
% - Solutions
%   - Parameters
%   - Eigenvalues
%   - HQM
% - Discussion and Conclusion

\section*{Abstract}
%
We introduced a method of optimizing four geometric parameters of a bicycle's
design to minimize the so called Handling Quality Metric
(HQM)~\cite{Moore2016}. Here we expand that method to optimize all X geometric
and inertial parameters of the benchmark parameterimization of the linear
Whipple-Carvallo bicycle model under constraints that guarantee a physically
realizable bicycle. This improves over the prior work by expanding the search
space with many more parameters and the guarntee of realizability.

\section{Introduction}
%
Physical design features of ground vehicles can affect their lateral handling
qualities. Geometry, mass, and mass distribution of the vehicle's primary
components as well as tire characteristics are primary contributors to poor and
good handling due to their important influence on the vehicle's dynamics. In
past work, we have presented a theoretical and computational framework for
assessing the lateral task-independent handling qualities of simplified single
track vehicle designs~\cite{Hess2012,Moore2012}. In subsequent work, we showed
that minimizing our proposed handling quality metric (HQM) can produce theoretically
optimal handling designs when only four geometric parameters are explored as
the optimization variables~\cite{Moore2016}. The present work's goal is to
expand this optimization problem to all of the geometry, mass, and inertial
parameters present in the linear Whipple-Carvallo bicycle model~\cite{Meijaard2007}.
This broadens the search space considerably but we constrain it so that only
realizable optimal bicycle designs are discovered. To do so, we formulate a
constrained optimization problem and use derivative-free optimization to
discover optimal, yet realizable, bicycle designs.

\section{Bicycle Model Parameterization}
%
Our problem formulation relies on a new bicycle model parameterization that
reflects both a reformulation of and addition to the benchmark parameterization
of the linear Whipple-Carvallo bicycle model~\cite{Meijaard2007}. We call this
the ``principal parameterizatoin'' as opposed to the ``benchmark
parameterization''. This parameterization differs from the benchmark
parameterization in three ways.  Firstly, the person and rear frame are treated
as separate rigid bodies each with their on inertial parameters. Secondly, we
express the inertial parameters of each rigid body in terms of central
principal radii of gyration to decouple the mass from the inertia terms.
Lastly, we introduce two simple dimensional parameters that define the
geometric extents of the person which are used to constrain the location of the
person's body. Table~\ref{tab:parameters} provides the parameters and the
reference values which are derived from the measurements of a Batavus Browser
Bicycle and the rider Jason presented in~\cite{Moore2012}. This
parameterization can be transformed into the benchmark parameterization
readily, but not vice versa.
%
\begin{table}
  \caption{Full set of 47 principal parameters and their default values derived
  from the measurements in~\cite{Moore2012}.}
  \label{tab:parameters}
  \centering
  \begin{tabular}{llll}
    \toprule
    Variable & Value & Units & Description \\
    \midrule
    $c$ &  0.068581 & \si{\meter} & Trail \\
    $w$ &  1.1210 & \si{\meter} & Wheelbase \\
    $\lambda$ &  0.39968 & \si{\radian} & Steer axis tilt \\
    $g$ &  9.81 & \si{\meter\per\second\squared} & Acceleration due to gravity \\
    $v$ &  3.0 & \si{\meter\per\second} & Forward speed \\
    Rear Wheel [R] & & \\
    \midrule
    $m_R$ &  3.11 & \si{\kilogram} & mass \\
    $r_R$ &  0.34096 & \si{\meter} & Rear wheel radius \\
    $x_R$ & 0 & \si{\meter} & Rear wheel mass center \\
    $y_R$ & 0 & \si{\meter} & Rear wheel mass center \\
    $z_R$ & -0.34096 & \si{\meter} & Rear wheel mass center \\
    $k_{Raa}$ &  0.17050 & \si{\meter} & Rear wheel central principal radii of gyration \\
    $k_{Rbb}$ &  0.17050 & \si{\meter} & Rear wheel central principal radii of gyration  \\
    $k_{Ryy}$ &  0.22136 & \si{\meter} & Rear wheel central principal radii of gyration \\
    Front Wheel [F] & & \\
    \midrule
    $m_F$ &  2.02 & \si{\kilogram} & Mass \\
    $r_F$ &  0.34353 & \si{\meter} & Radius \\
    $x_F$ &  1.1210 & \si{\meter} & Mass center \\
    $y_F$ & 0.0 & \si{\meter} & Mass center \\
    $z_F$ & -0.34353 & \si{\meter} & Mass center \\
    $k_{Faa}$ &  0.20917 & \si{\meter} & Central principal radius of gyration \\
    $k_{Fbb}$ &  0.20917 & \si{\meter} & Central principal radius of gyration \\
    $k_{Fyy}$ &  0.27179 & \si{\meter} & Central principal radius of gyration \\
    Person [P] & & \\
    \midrule
    $l_P$ &  1.7280 & \si{\meter} & Body length \\
    $w_P$ &  0.48300 & \si{\meter} & Body width \\
    $m_P$ &  83.500 & \si{\kilogram} & Mass \\
    $x_P$ &  0.31577 & \si{\meter} & Mass center \\
    $y_P$ & 0.0 & \si{\meter} & Mass center \\
    $z_P$ & -1.0990 & \si{\meter} & Mass center \\
    $k_{Paa}$ &  0.36797 & \si{\meter} & Central principal radius of gyration \\
    $k_{Pbb}$ &  0.15276 & \si{\meter} & Central principal radius of gyration \\
    $k_{Pyy}$ &  0.36717 & \si{\meter} & Central principal radius of gyration \\
    $\alpha_P$ & 0.18618 & \si{\radian} & Principal axis angle \\
    Front Frame [H] & & \\
    \midrule
    $m_H$ & 3.2200 & \si{\kilogram} & Mass \\
    $x_H$ & 0.86695 & \si{\meter} & Mass center \\
    $y_H$ & 0.0 & \si{\meter} & Mass center \\
    $z_H$ & -0.74824 & \si{\meter} & Mass center \\
    $k_{Haa}$ & 0.29556 & \si{\meter} & Central principal radius of gyration \\
    $k_{Hbb}$ & 0.14493 & \si{\meter} & Central principal radius of gyration \\
    $k_{Hyy}$ & 0.27630 & \si{\meter} & Central principal radius of gyration \\
    $\alpha_H$ & 0.36995 & \si{\radian} & Principal axis angle \\
    Rear Frame [D] & & \\
    \midrule
    $m_D$ &  9.8600 & \si{\kilogram} & Mass \\
    $x_D$ &  0.27595 & \si{\meter} & Mass center \\
    $y_D$ & 0 & \si{\meter} & Mass center \\
    $z_D$ & -0.53784 & \si{\meter} & Mass center \\
    $k_{Daa}$ &  0.28587 & \si{\meter} & Central principal radius of gyration \\
    $k_{Dbb}$ &  0.22079 & \si{\meter} & Central principal radius of gyration \\
    $k_{Dyy}$ &  0.36539 & \si{\meter} & Central principal radius of gyration \\
    $\alpha_D$ &  1.1722 & \si{\radian} & Principal axis angle \\
    \bottomrule
  \end{tabular}
\end{table}

\section{Bounds and Constraints}
%
The optimal principal parameters are subject to a set of constraints designed
to ensure that a physically realizable bicycle is obtained form the opimization
procedure. These constraints are made up of bounds on the free parameters and
both equality and inequality constraints among the parameters. Below the basic
constraint concepts presented and grouped by the associated rigid body or
collection thereof:
%
\begin{description}
  \item[Total] The resulting combination of the five rigid bodies.
    \begin{itemize}
      \item The physical extents of the rigid bodies must exist above the
        ground plane.
      \item Both bicycle and rider are symmetric about the rider's sagittal
        plane.
      \item The mass of each bicycle rigid body is positive, greater than a
        minimum value, and the total mass is below a reasonably lift-able
        amount.
      \item The wheels cannot overlap.
      \item The bicycle cannot topple forward during hard braking or backward
        during hard acceleration.
    \end{itemize}

  \item[Wheels] Both front and rear wheels have identical constraints.
    \begin{itemize}
      \item Wheel radius and mass must be greater than a minimum value.
      \item Wheels are inertially wheel-like, i.e. symmetric about each plane
        and most of the mass is at the rim.
    \end{itemize}

  \item[Frames] Front frame (handlebar + fork) and rear frame
    \begin{itemize}
      \item The mass and inertia of the frames are positive and large enough to
        be constructed from materials in a space frame of specified minimal
        density.
      \item The rear frame is planar in nature and the front frame's moments of
        inertia are consitently dependent.
    \end{itemize}

  \item[Rider] A single rigid body represents the rider.
    \begin{itemize}
      \item Rider mass is that of an average person.
      \item The rider's joint angles are fixed in a nominal configuration
        typical of upright bicycling and the resulting mass distribution is
        derived from standard body segment estimation methods.
    \end{itemize}

\end{description}

These bounds, equality, and inequality constraints are presented mathematically
in tables \ref{tab:bounds}, \ref{tab:equality-constraints},
\ref{tab:inequality-constraints} respectively and explained in more detail in
the following sections.
%
\begin{table}
  \caption{Free parameter upper and lower bounds}
  \label{tab:bounds}
  \centering
  \begin{tabular}{ccccc}
    \toprule
    Min & & Parmeter & & Max\\
    \midrule
    $-\infty$ & $\leq$ & $w       $  &  $\leq$  &  $\infty$ \\
    $-\infty$ & $\leq$ & $c       $  &  $\leq$  &  $\infty$ \\
    $-\pi/2 $ & $\leq$ & $\lambda $  &  $\leq$  &  $\pi/2$ \\
    $1.0    $ & $\leq$ & $m_D     $  &  $\leq$  &  $\infty$ \\
    $-\infty$ & $\leq$ & $x_D     $  &  $\leq$  &  $\infty$ \\
    $-\infty$ & $\leq$ & $z_D     $  &  $\leq$  &  $0.0$    \\
    $0.0    $ & $\leq$ & $k_{Daa} $  &  $\leq$  &  $\infty$ \\
    $0.0    $ & $\leq$ & $k_{Dbb} $  &  $\leq$  &  $\infty$ \\
    $-\pi/2 $ & $\leq$ & $\alpha_D$  &  $\leq$  &  $\pi/2$  \\
    $-\infty$ & $\leq$ & $x_P     $  &  $\leq$  &  $\infty$ \\
    $-\infty$ & $\leq$ & $z_P     $  &  $\leq$  &  $0.0$    \\
    $-\pi/2 $ & $\leq$ & $\alpha_P$  &  $\leq$  &  $\pi/2$ \\
    $0.25   $ & $\leq$ & $m_H     $  &  $\leq$  &  $\infty$ \\
    $-\infty$ & $\leq$ & $x_H     $  &  $\leq$  &  $\infty$ \\
    $-\infty$ & $\leq$ & $z_H     $  &  $\leq$  &  $0.0$ \\
    $0.0    $ & $\leq$ & $k_{Haa} $  &  $\leq$  &  $\infty$ \\
    $0.0    $ & $\leq$ & $k_{Hbb} $  &  $\leq$  &  $\infty$ \\
    $0.0    $ & $\leq$ & $k_{Hyy} $  &  $\leq$  &  $\infty$ \\
    $-\pi/2 $ & $\leq$ & $\alpha_H$  &  $\leq$  &  $\pi/2$ \\
    $0.127  $ & $\leq$ & $r_R     $  &  $\leq$  &  $\infty$ \\
    $1.0    $ & $\leq$ & $m_R     $  &  $\leq$  &  $\infty$ \\
    $0.127  $ & $\leq$ & $r_F     $  &  $\leq$  &  $\infty$ \\
    $1.0    $ & $\leq$ & $m_F     $  &  $\leq$  &  $\infty$ \\
    \bottomrule
  \end{tabular}
\end{table}
%
\begin{table}
  \caption{Equality constraints}
  \label{tab:equality-constraints}
  \centering
  \begin{tabular}{lll}
    \toprule
    Constraint & Equation & Description \\
    \midrule
    $g_1$ & $I_{Dyy} = \sqrt{I_{Dxx}^2 + I_{Dzz}^2}$ & Rear frame is planar. \\
    $g_2$ & $k_{Ryy} = r_R$  & Rear wheel is a ring \\
    $g_3$ & $k_{Raa} = k_{Ryy}/2$ & Rear wheel is a ring \\
    $g_4$ & $k_{Rbb} = k_{Ryy}/2$ & Rear wheel is a ring \\
    $g_5$ & $k_{Fyy} = r_F$  & Front wheel is a ring \\
    $g_6$ & $k_{Faa} = k_{Fyy}/2$ & Front wheel is a ring \\
    $g_7$ & $k_{Fbb} = k_{Fyy}/2$ & Front wheel is a ring \\
    \bottomrule
  \end{tabular}
\end{table}
%
\begin{table}
  \caption{Inequality constraints}
  \label{tab:inequality-constraints}
  \centering
  \begin{tabular}{lll}
    \toprule
    Constraint & Equation & Description \\
    \midrule
    $c_1$    & $\sqrt{I_{Hxx}^2+I_{Hzz}^2} \geq I_{Hyy}$           & Consistent moments of inertia. \\
    $c_2$    & $0 \geq z_P + l_P/2 \cos{\alpha_P}$                 & Person cannot penetrate ground. \\
    $c_3$    & $0 \geq z_P + w_P/2 \sin{\alpha_P}$                 & Person cannot penetrate ground. \\
    $c_4$    & $0 \geq z_P - l_P/2 \cos{\alpha_P}$                 & Person cannot penetrate ground. \\
    $c_5$    & $0 \geq z_P - w_P/2 \sin{\alpha_P}$                 & Person cannot penetrate ground. \\
    $c_6$    & $x_T \geq |z_T|/4$                                  & Maximum acceleration of $1/4g$. \\
    $c_7$    & $w-x_T \geq 3/4|z_T|$                               & Maximum deceleration of $3/4g$. \\
    $c_8$    & $2k_{Hyy} \geq \sqrt{(x_H - w)^2 + (z_H + r_F)^2}$  & Minimal inertial spread. \\
    $c_9$    & $2k_{Dyy} \geq \sqrt{(x_D - 0)^2 + (z_D + r_R)^2}$  & Minimal inertial spread. \\
    $c_{10}$ & $w \geq r_F +r_R$                                   & Non-overlapping wheels. \\
    $c_{11}$ & $25\si{\kg} \geq m_D + m_H + m_R + m_F$             & Maximum bicycle mass. \\
    $c_{12}$ & $-z_D \geq 1.4 k_{Dyy}$                             & Rear frame cannot penetrate ground. \\
    $c_{13}$ & $-z_H \geq 1.4 k_{Hyy}$                             & Front frame cannot penetrate ground. \\
    $c_{14,\ldots,21}$ & $0 \geq s_1,\ldots,s_8$                   & Closed loop stability. \\
    \bottomrule
  \end{tabular}
\end{table}

\subsubsection*{Rear Frame [D]}
%
Several constraints are set for the rear frame. We constrain the rear frame to
be planar, $g_1$, and symmetric with respect to the $XZ$ plane. We prevent the
rear frame from penetrating the ground by limiting the inertial spread with
respect to its mass center, $c_{12}$, but also set a minimum inertial spread to
ensure a frame can span from the rear wheel to the mass center of the rear
frame, $c_9$. The spread factor in $c_{12}$ of 1.4 is based on the ratio of
geometrical spread of a typical bicycle frame and its radius of gyration.
Several parameters are bounded. We require the rear frame mass to be positive,
the center of mass not penetrate the ground, and we allow for any angular
orientation of the $XZ$ principal directions but limit the angle to
$-\frac{\pi}{2} \leq \alpha_D \leq \frac{\pi}{2}$ as angle beyond that are
redundant.

\subsection{Person [P]}

We assume that the person's joint configurations are such that they are in a
nominal configuration for pedaling, i.e. an average normal everyday riding
position on a typical bicycle, i.e. they stayed in the same configuration as
they were seated on the Batavus Browser bicycle. The person is assumed to be
symmetric about the $XZ$ plane. We allow the rider to be rotated about the $Y$
axis and positioned anywhere within the plane of symmetry above the ground.

To prevent the rider from being positioned and oriented such that their body is
not penetrating the the ground we introduce two dimensions that define a cross
whose apex is at the center of mass of the person and the cross axes are
parallel to the principal axes in the $XZ$ plane. $l_P / 2$ is the distance
along the principal axis to the tip of the toes and $w_P$ is the distance along
the second principal axes to the tip of the hands. Constraints
$c_2,\ldots,c_5$.

\subsection{Front Frame}
%
The front frame is symmetric about the $XZ$ plane so $I_{Hxy}, I_{Hyz} = 0$. We
allow for any angular orientation of the $XZ$ principal directions but limit
the angle to $-\frac{\pi}{2} \leq \alpha_H \leq \frac{\pi}{2}$.  We prevent the
rear frame from penetrating the ground by limiting the inertial spread with
respect to its mass center, $c_{13}$, but also set a minimum inertial spread to
ensure a frame can span from the rear wheel to the mass center of the rear
frame, $c_8$. The spread factor in $c_{13}$ of 1.4 is based on the ratio of
geometrical spread of a typical bicycle frame and its radius of gyration. The
front frame is not planar but is narrow with respect to the $XZ$ plane, which
is enforced by constraint $c_1$.

\subsection{Front [F] and Rear [R] Wheels}
%
We enforce the assumption that both wheels have moments of inertia of that of a
simple ring, $g_2 \ldots g_7$ and that mass and radius should be greater than a
minimal size based on small purchable spoked wheel with tire.

\subsection{Total Bike [T]}
%
The trail and wheelbase can take on any real values. The steer axis tilt is
limted to 180 degrees. We introduce a constraint $c_{10}$ that prevents the
wheels from physically overlapping and require that the bicycle be liftable by
an average person, $c_{11}$. Finally, we require that the bicycle not topple
forward during hard breaking or backward during hard acceleration.

\begin{align}
  -\frac{3g}{4} < \textrm{acceleration} < \frac{g}{4}
\end{align}

This translates to two constraints, $c_6,c_7$ that bound the total center of
mass $(x_T,z_T)$ in a triangle in the $XZ$ plane. Lastly, we constrain the
eight closed loop eigenvalues associated with the controller in
~\cite{Hess2012} to be stable, i.e. have negative real parts.  These are
expressed in constraints $c_{14},\ldots,c_{21}$. Closed loop stability is
required for the HQM to provide a meaningful result.

\section{Free parameters}
%
The above constraints leaves 23 of the 47 parameters free for optimizing which
we collect in the vector $\mathbf{p}\in\mathbb{R}^{23}$ and define as:
%
\begin{align}
  \begin{split}
    \mathbf{p} = [ &
       w        \quad
       c        \quad
       \lambda  \quad
       m_D      \quad
       x_D      \quad
       z_D      \quad
       k_{Daa}  \quad
       k_{Dbb}  \quad
       \alpha_D \quad
       x_P      \quad
       z_P      \quad
       \alpha_P \\
     & m_H      \quad
       x_H      \quad
       z_H      \quad
       k_{Haa}  \quad
       k_{Hbb}  \quad
       k_{Hyy}  \quad
       \alpha_H \quad
       r_R      \quad
       m_R      \quad
       r_F      \quad
       m_F]
  \end{split}
\end{align}

\section{Optimization}
%
Our objective in the optimization is to minimize the peak HQM value subject to
the bounds, $\mathbf{p}^L,\mathbf{p}^U$, and the constraints
$\mathbf{g}(\mathbf{p}),\mathbf{c}(\mathbf{p})$ . Given a set of bicycle model
parameter values we generate a bandwidth limited human-like controller using
the methods in ~\cite{Moore2012}. Once the closed loop stable controller is
constructed, the HQM can be computed as per the definition in \cite{Hess2012}
and the scaler peak value returned as the objective $J$. This problem is
presented as a non-linear programming problem in the following equation.
%
\begin{equation}
  \begin{aligned}
    & \underset{\mathbf{p}}{\text{minimize}} & & J(\mathbf{p})=max(\textrm{HQM}(\mathbf{p})) \\
    & \text{subject to} & & \\
    & & & \mathbf{g}(\mathbf{p}) \leq \mathbf{0} \\
    & & & \mathbf{c}(\mathbf{p}) = \mathbf{0} \\
    & & & \mathbf{p}^L \leq \mathbf{p} \leq \mathbf{p}^U
  \end{aligned}
\end{equation}

We make use of the derivative-free optimizer CMA-ES~\cite{Hansen1996} to find
solutions to this problem. The optimization supports parameter bounds and
equality constraints but does not support inequality constraints. To get around
this limitation we move the inequality constraints into the objective function
an penalize the objective if the constraints are violated with the following
rules:
%
\begin{align}
  J(\mathbf{p}) =
  \begin{cases}
    max(\textrm{HQM}(\mathbf{p})) & \textrm{if} \quad any(\mathbf{g}(\mathbf{p})) < 0 \\
    30 + ||\mathbf{g}_{+}(\mathbf{p})||/10 & \textrm{if} \quad
      any(\mathbf{g}(\mathbf{p})) \geq 0 \textrm{ and } ||\mathbf{g}_{+}(\mathbf{p})|| < 30 \\
    ||\mathbf{g}_{+}(\mathbf{p})|| & \textrm{if} \quad
      any(\mathbf{g}(\mathbf{p})) \geq 0 \textrm{ and } ||\mathbf{g}_{+}(\mathbf{p})|| \geq 30
  \end{cases}
\end{align}
where $||\mathbf{g}_{+}||$ is the norm of the positive elements of
$\mathbf{g}$.

This creates a discontinous objective function but in practice the CMS-ES
algorthim is able to move into the parameter space where all the constraints
are satisfied and find a (local) minima. For our purposes, this sufficiently
finds parameter values that produce an optimally handling design.

\section{Results}
%
We discover four bicycles for four different design speeds (3, 5, 7, and 9 m/s)
that have an optimally low HQM, see Table~\ref{tab:hqm} and satisify all
constraints and parameter boundaries. We belive these bicycles to be physically
realizable with minor differences. The most striking feature is that all of the
bicycle are larger that the reference bicycle in some way.  The 3 m/s bicycle
places both the rider's mass center and the rear frame's mass center close to 3
meters above the ground plane. The 5 and 7 m/s bicycles have wheelbase values
that are simlar to the reference bicycle, but 3 and 9 m/s have very large wheel
bases. The 9 m/s bicycle is very large overall. Only the 5 m/s bicycle is of a
scale close to a typical bicycle. It is noteable that the 9 m/s bicycle has
signficant negative trail.

3
- Rider is over 3 m in the air and leaned forward.
- 3 m wheel base with 0.5 meter positive trail
- wheels are over double the diameter of browser wheels

5 m/s
- rider is laid back a 0.5 meter off ground
- < 1 m wheelbase

7 m/s
- miminal rotatoinal inertia about the steer axis
- tall slender fork/handlebar
- no trail
- half sized wheels (~12 inches)
- normal wheelbase
- rider tipped forward and low (0.6 m)
- vertical steer axis
- 3 m high frame center of mass
%
\begin{figure}
  \centering
  \includegraphics[width=6in]{../figures/optimal-geometries.png}
  \caption{Depcitions of the bicycle geometry and geometric representations of
    the inertial quanties for the reference bicycle and four optimal solutions
    at 3, 5, 7, and 9 m/s. Five rigid bodies are shown for each bicycle: front
    wheel (oragne), rear wheel (purple), rear frame (blue), front frame
    (green), and person (red). The solid black lines represent the essential
    bicycle geometry. The dotted black line represents the steer axis. The
    solid colored curves represent the contours of solid ellipsoids with
    equivalent inertia as the principal inertia of the associated rigid body.
    The dotted colored lines represent the extents of the centerial radii of
    gyration of each rigid body.}
\end{figure}
%
\begin{table}
  \caption{Peak HQM values for the reference bicycle and the optimal bicycles
    at each speed.}
  \label{tab:hqm}
  \centering
  \begin{tabular}{llll}
    \toprule
    Speed [m/s] & Reference Peak HQM & Optimal Peak HQM & Percent Improvement \\
    \midrule
    3 & 13.0753 & 2.0118 & 85\% \\
    5 & 4.5213  & 0.0115 & 100\% \\
    7 & 3.0434  & 0.0220 &  99\% \\
    9 & 2.3377  & 0.8386 &  64\% \\
    \bottomrule
  \end{tabular}
\end{table}
%
\begin{figure}
  \centering
  \includegraphics[width=6in]{../figures/optimal-eigenvalues.png}
  \caption{}
\end{figure}
%
\begin{table}
  \caption{Optimal principal parameter values for each design speed.}
  \label{tab:optimal-values}
  \centering
  \begin{tabular}{lllll}
    \toprule
    Parameter & 3 m/s & 5 m/s & 7 m/s & 9 m/s \\
    \midrule
alphaD & 0.66300  & 0.91544    & 1.1228    & 1.076700 \\
alphaH & 0.57150  & -1.2647 &    1.5708    & 0.23760 \\
alphaP & -0.66130 & 1.1497 &     -0.83627  & 1.5584 \\
c      & 0.68780  & -0.0050384 & 0.0010500 & -0.48397 \\
g      & 9.8100   & 9.8100 &     9.8100 &    9.8100 \\
kDaa   & 1.4491   & 1.2179 &     1.3242 &    0.86084 \\
kDbb   & 0.72360  & 1.3868 &     1.2903 &    2.0139 \\
kDyy   & 1.4711   & 1.5585 &     1.5550 &    2.0305 \\
kFaa   & 0.35495  & 0.12585 &    0.063500 &  1.0314 \\
kFbb   & 0.35495  & 0.12585 &    0.063500 &  1.0314 \\
kFyy   & 0.70990  & 0.25170 &    0.12700 &   2.0628 \\
kHaa   & 0.18620  & 0.049106 &   0.0 &       2.8463 \\
kHbb   & 1.2578   & 1.1451 &     2.0055 &    0.16848 \\
kHyy   & 0.63610  & 0.38324 &    0.81835 &   0.66922 \\
kPaa   & 0.36797  & 0.36797 &    0.36797 &   0.36797 \\
kPbb   & 0.15276  & 0.15276 &    0.15276 &   0.15276 \\
kPyy   & 0.36717  & 0.36717 &    0.36717 &   0.36717 \\
kRaa   & 0.47910  & 0.25306 &    0.073095 &  1.5135 \\
kRbb   & 0.47910  & 0.25306 &    0.073095 &  1.5135 \\
kRyy   & 0.95820  & 0.50612 &    0.14619 &   3.0270 \\
lP     & 1.7280   & 1.7280 &     1.7280 &    1.7280 \\
lam    & -0.21250 & 0.27137 &    -0.027560 & 0.23941 \\
mD     & 3.2233   & 11.534 &     10.315 &    7.0223 \\
mF     & 1.6211   & 4.4038 &     2.6441 &    5.5058 \\
mH     & 0.25000  & 0.25000 &    0.49309 &   3.5435 \\
mP     & 83.500   & 83.500 &     83.500 &    83.500 \\
mR     & 12.625   & 8.3601 &     6.3080 &    2.2534 \\
rF     & 0.70990  & 0.25170 &    0.12700 &   2.0628 \\
rR     & 0.95820  & 0.50612 &    0.14619 &   3.0270 \\
v      & 3.0      & 5.0 &        7.0 &       9.0 \\
w      & 2.8655   & 0.84711 &    1.1570 &    5.5573 \\
wP     & 0.48300  & 0.48300 &    0.48300 &   0.48300 \\
xD     & 0.95820  & 0.28749 &    0.42710 &   1.0204 \\
xF     & 2.8655   & 0.84711 &    1.1570 &    5.5573 \\
xH     & 2.3561   & 0.53227 &    1.1449 &    5.6145 \\
xP     & 0.76460  & 0.27642 &    0.52614 &   1.0791 \\
xR     & 0.0      & 0.0 &        0.0 &       0.0 \\
yD     & 0.0      & 0.0 &        0.0 &       0.0 \\
yF     & 0.0      & 0.0 &        0.0 &       0.0 \\
yH     & 0.0      & 0.0 &        0.0 &       0.0 \\
yP     & 0.0      & 0.0 &        0.0 &       0.0 \\
yR     & 0.0      & 0.0 &        0.0 &       0.0 \\
zD     & -2.6050  & -2.7190 &    -2.9764 &   -4.0653 \\
zF     & -0.70990 & -0.25170 &   -0.12700 &  -2.0628 \\
zH     & -1.2977  & -0.78065 &   -1.3399 &   -1.7755 \\
zP     & -3.1937  & -0.45256 &   -0.58632 &  -2.6623 \\
zR     & -0.95820 & -0.50612 &   -0.14619 &  -3.0270 \\
    \bottomrule
  \end{tabular}
\end{table}

\section{Discussion and Conclusion}
%

\bibliographystyle{acm}
\bibliography{BMD2019}

\end{document}
