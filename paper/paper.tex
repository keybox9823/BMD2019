\documentclass{icsc}

\usepackage{siunitx}

\begin{document}

\begin{center}
  \fontsize{14}{20}{\bf Optimal bicycle design to maximize handling and safety}
\end{center}

\begin{center}
  \normalsize{\bf{Jason K. Moore, Mont Hubbard,  Ronald A. Hess}}
\end{center}

\begin{center}
  \begin{tabular}{c}
    Department of Mechanical and Aerospace Engineering\\
    University of California, Davis\\
    One Shields Avenue, Davis, CA, USA 95616\\
    e-mail: jkm@ucdavis.edu, mhubbard@ucdavis.edu, rahess@ucdavis.edu\\
  \end{tabular}
\end{center}

\begin{keywords}
  bicycle,
  design,
  handling qualities,
  control,
  optimization
\end{keywords}

\appendix

\section{Free Parameters and Constraints}

\subsection{Rear Frame [D]}

There are six free parameters associated with the rear frame's inertial characteristics:

\begin{description}
  \item[$m_D$] Mass of the rear frame.
  \item[$k_{Dxx}$] Central radius of gyration with respect to the $X$ axis where $k_{Dxx}^2=I_{Dxx} / m_D$.
  \item[$k_{Dyy}$] Central radius of gyration with respect to the $Y$ axis where $k_{Dyy}^2=I_{Dyy} / m_D$.
  \item[$k_{Dzz}$] Central radius of gyration with respect to the $Z$ axis where $k_{Dzz}^2=I_{Dzz} / m_D$.
  \item[$\alpha_D$] Angle between the $X$ axis and the primary principal inertia axis.
  \item[$x_D$] Location along the $X$ axis of the rear frame center of mass.
  \item[$z_D$] Location along the $Z$ axis of the rear frame center of mass.
\end{description}

\begin{itemize}
  \item The body is symmetric with respect to the $XZ$ plane, i.e. $I_{Dxy} = I_{Dyz} = 0$.
  \item We require that the center of mass not penetrate the ground, i.e. $z_D < 0$.
  \item We require that the mass be positive, i.e. $m_D > 0$.
  \item We allow for any angular orientation of the $XZ$ principal directions but limit the angle to $-\frac{\pi}{2} \leq \alpha_D \leq \frac{\pi}{2}$.
  \item We enforce that the body is planar with $k_{Dyy} = \sqrt{k_{Dxx}^2 + k_{Dyy}^2}$.
\end{itemize}

\subsection{Person [P]}

We assume that the person's joint configurations are such that they are in a nominal configuration for pedaling, i.e. an average normal everyday ???or maximum power?? riding position. The person is assumed to be symmetric about the $XZ$ plane. There are four fixed parameters that enforce this assumption: 

\begin{description}
  \item[$m_P$] The mass of the person which is fixed at $XX \si{kg}$.
  \item[$k_{Pxx}$] Central radius of gyration with respect to the $X$ axis where $I_{Pxx} = m_P k_{Pxx}^2$.
  \item[$k_{Pyy}$] Central radius of gyration with respect to the $Y$ axis where $I_{Pyy} = m_P k_{Pyy}^2=$.
  \item[$k_{Pzz}$] Central radius of gyration with respect to the $Z$ axis where $I_{Pzz} = m_P k_{Pzz}^2=$.
  \item[$I_{Bxy}, I_{Byz}$] Products of inertia are zero.
  \item[$y_P$] Is zero.
\end{description}

We allow the rider to be rotated about the $Y$ axis and positioned anywhere within the plane of symmetry. Three free parameters are associated with these assumptions:
  
\begin{description}
  \item[$\alpha_P$] Angle between the $X$ axis and the primary principal inertia axis of the person.
  \item[$x_P$] Location along the $X$ axis of the person's center of mass.
  \item[$z_P$] Location along the $Z$ axis of the person's center of mass.
\end{description}

To prevent the rider from being positioned and oriented such that their body is not penetrating the the ground we introduce two dimensions that define a cross whose apex is at the center of mass of the person and the cross axes are parallel to the principal axes in the $XZ$ plane. $l_P / 2$ is the distance along the principal axis to the tip of the toes and $w_P$ is the distance along the second principal axes to the tip of the hands.

\begin{align}
  z_P + l_P / 2 \cos{\alpha_P} & \leq 0 \\
  z_P + w_P / 2 \sin{\alpha_P} & \leq 0 \\
  z_P - l_P / 2 \cos{\alpha_P} & \leq 0 \\
  z_P - w_P / 2 \sin{\alpha_P} & \leq 0
\end{align}

\subsection{Rear Wheel [R]}

We enforce the assumption that the wheel should be wheel-like, i.e. $I_{Ryy} = m_R r_R^2$ and $I_{Rxx} = I_{Rzz} = I_{Ryy} / 2$ thus there are two free parameters:

\begin{description}
  \item[$r_R$] Outer radius of the rear wheel.
  \item[$m_R$] Mass of the rear wheel.
\end{description}

We also enforce that the mass and radius should be positive with:

\begin{align}
  m_R \geq 0 \\
  r_R \geq 0
\end{align}

\subsection{Front Wheel}

Same as rear wheel.

\subsection{Front Frame}

The front frame is symmetric about the $XZ$ plane so $I_{Hxy}, I_{Hyz} = 0$. There are seven free parameters:

\begin{description}
  \item[$m_H$] The mass of the rear frame.
  \item[$k_{Hxx}$] Central radius of gyration with respect to the $X$ axis where $k_{Hxx}^2=I_{Hxx} / m_H$.
  \item[$k_{Hyy}$] Central radius of gyration with respect to the $X$ axis where $k_{Hyy}^2=I_{Hyy} / m_H$.
  \item[$k_{Hzz}$] Central radius of gyration with respect to the $X$ axis where $k_{Hzz}^2=I_{Hzz} / m_H$.
  \item[$\alpha_H$] Angle between the $X$ axis and the primary principal inertia axis.
  \item[$x_B$] Location along the $X$ axis of the rear frame's center of mass.
  \item[$z_H$] Location along the $Z$ axis of the rear frame's center of mass.
\end{description}




\subsection{Total Bike}

There are three remaing free parameters associated with the fundamental geometry of the bicycle:

\begin{description}
  \item[$w$] Wheelbase
  \item[$c$] Trail
  \item[$\lambda$] Steer axis tilt
\end{description}

To prevent the wheels from overlapping we introduce:

\begin{align}
  w > r_F + r_R
\end{align}

The trail can take on any real value as well as the steer axis tilt except the bound the angle to a minimal range:

\begin{align}
  -\frac{\pi}{2} \leq \lambda \leq \frac{\pi}{2}
\end{align}

We require that the bicycle be liftable by an average person with:

\begin{align}
  m_T = m_H + m_B + m_F + m_R \leq 25 \si{kg}
\end{align}

Finally, we require that the bicycle not topple forward during hard breaking and backward during hard acceleration where:

\begin{align}
  -g < a < \frac{g}{4}
\end{align}

This translates to two constraints that bound the total center of mass in a triangle in teh $XZ$ plane:

\begin{align}
  \frac{x_T}{|z_T|} < \frac{1}{4} \\
  \frac{x_T - w}{|z_T|} < -1
\end{align}

\subsection{Summary}

The above definitions represent a constrained optimization in 23 free parameters with 17 constraints. This bounds the space of bicycle designs to be physically realizable and excludes bicycles that don't have the inherent characteristics of a bicycle.

\begin{align}
  p = & [m_D, k_{Dxx}, k_{Dzz}, \alpha_D, x_D, z_D, x_P, z_P, \alpha_P, r_R, m_R, r_F, m_H, \\
      & x_H, z_H, k_{Hxx}, k_{Hyy}, k_{Hzz}, \alpha_H, w, c, \lambda]
\end{align}

with these fixed values:

\begin{align}
  I_{Dxy} = I_{Dyz} = 0 \\
  % 
  l_P = ? \si{m}\\
  w_P = ? \si{m} \\
  m_P = 83.5 \si{kg} \\
  y_P = 0 \si{\meter} \\
  I_{Pxx} = 10.98524 \si{\kg\meter^2} \\
  I_{Pyy} = 11.2571 \si{\kg\meter^2}\\
  I_{Pzz} = 2.26922 \si{\kg\meter^2}\\
  I_{Pxy} = I_{Pyz} = 0 \si{\kg\meter^2}\\
  I_{Pxz} = -1.70211 \si{\kg\meter^2}\\
  %
  I_{Bxy} = I_{Byz} = 0 \\
  % 
  I_{Hxy} = I_{Hyz} = 0 \\
  %
  I_{Rxy} = I_{Ryz} = I_{Rxz} = 0 \\
  %
  I_{Fxy} = I_{Fyz} = I_{Fxz} = 0 \\
\end{align}

with these parameter bounds:

\begin{align}
  m_R > 0 \\
  m_F > 0 \\
  m_H > 0 \\
  m_D > 0 \\
  r_R > 0 \\
  r_F > 0 \\
  z_H < 0 \\
  z_D < 0 \\
  -\frac{\pi}{2} \leq \alpha_H \leq \frac{\pi}{2} \\
  -\frac{\pi}{2} \leq \alpha_P \leq \frac{\pi}{2} \\
  -\frac{\pi}{2} \leq \alpha_D \leq \frac{\pi}{2} \\
  -\frac{\pi}{2} \leq \lambda \leq \frac{\pi}{2}
\end{align}

with these parameter convertions:

\begin{align}
  I_{Dxx} = m_D k_{Dxx}^2 \\
  I_{Dzz} = m_D k_{Dzz}^2 \\
  I_{Dyy} = \sqrt{I_{Dxx}^2 + I_{Dzz}^2} \\
  I_{Dxz} = f(I_{Dxx}, I_{Dzz}, \alpha_D)  \\
  %
  m_B = m_D + m_P \\
  x_B = \frac{m_D x_D + m_P x_P}{m_B} \\
  z_B = \frac{m_D z_D + m_P z_P}{m_B} \\
  I_{Bxx} = f(m_B, I_P, I_D, x_D, z_D, x_P, z_P, x_B, z_B) \\
  I_{Byy} = f(m_B, I_P, I_D, x_D, z_D, x_P, z_P, x_B, z_B) \\
  I_{Bzz} = f(m_B, I_P, I_D, x_D, z_D, x_P, z_P, x_B, z_B) \\
  I_{Bxy} = 0 \\
  I_{Byz} = 0 \\
  I_{Bxz} = f(m_B, I_P, I_D, x_D, z_D, x_P, z_P, x_B, z_B) \\
  % 
  I_{Hxx} = m_H k_{Hxx}^2 \\
  I_{Hzz} = m_H k_{Hzz}^2 \\
  I_{Hyy} = \sqrt{I_{Hxx}^2 + I_{Hzz}^2} \\
  I_{Hxz} = f(I_{Hxx}, I_{Hzz}, \alpha_H)  \\
  %
  I_{Ryy} = m_R r_R^2 \\
  I_{Rxx} = I_{Rzz} = I_{Ryy} / 2 \\
  %
  I_{Fyy} = m_F r_F^2 \\
  I_{Fxx} = I_{Fzz} = I_{Fyy} / 2 \\
  %
  m_T = m_H + m_B + m_F + m_R \\
  x_T = \frac{m_H x_H + m_B x_B + m_F w}{m_T} \\
  z_T = \frac{m_H z_H + m_B z_B + m_F r_F + m_R r_R}{m_T} 
\end{align}

and these constraints:

\begin{align}
  I_{Hyy} \leq \sqrt{I_{Hxx}^2 + I_{Hzz}^2} \\
  z_P + l_P / 2 \cos{\alpha_P} & \leq 0 \\
  z_P + w_P / 2 \sin{\alpha_P} & \leq 0 \\
  z_P - l_P / 2 \cos{\alpha_P} & \leq 0 \\
  z_P - w_P / 2 \sin{\alpha_P} & \leq 0 \\
  w > r_F + r_R \\
  m_T \leq 25 \si{kg} \\
  \frac{x_T}{|z_t|} < \frac{1}{4} \\
  \frac{x_T - w}{|z_T|} < -1
\end{align}

\end{document}
