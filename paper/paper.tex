\documentclass{bmd2019a}

\usepackage{siunitx}
\usepackage{amsmath}
\usepackage{booktabs}

\begin{document}

\begin{flushleft}
{\fontsize{16pt}{20pt}\selectfont%
  Expanded Optimization for Discovering\\}
{\fontsize{16pt}{20pt}\selectfont%
  Optimal Lateral Handling Bicycles}
\end{flushleft}

%%%%%%%%%%%%%%%% authors %%%%%%%%%%%%%%%
\begin{flushleft}
  {\fontsize{12}{14}{Jason K. Moore, Mont Hubbard, and Ronald A. Hess}\\}
  \textit{Mechanical and Aerospace Engineering\\
          University of California, Davis\\
          One Shields Avenue, Davis, CA, USA 95817\\
          e-mail: jkm@ucdavis.edu, mhubbard@ucdavis.edu, rahess@ucdavis.edu}
\end{flushleft}

% - Abstract
% - Introduction
% - Problem Definition
%   - Optimization
%   - Parameterization
%   - Bounds
%   - Constraints
% - Solutions
%   - Parameters
%   - Eigenvalues
%   - HQM
% - Discussion and Conclusion

\section*{Abstract}
%
Physical design features of ground vehicles can affect their lateral handling
qualities. Geometry, mass, and mass distribution of the vehicle's primary
components as well as tire characteristics are primary contributors to poor and
good handling due to their important influence on the vehicle's dynamics. In
past work, we have presented a theoretical and computational framework for
assessing the lateral task-independent handling qualities of simplified single
track vehicle designs~\cite{Hess2012,Moore2012}. In subsequent work, we showed
that minimizing our handling quality metric (HQM) can produce theoretically
optimal handling designs when only four geometric parameters are explored as
the optimization variables~\cite{Moore2016}. The present work's goal is to
expand this optimization problem to 24 geometry, mass, and inertial parameters
as the optimization variables so that a broader search space of optimal bicycle
designs is considered. This task is complicated by the fact that we desire to
limit the search space to physically realizable designs. Doing so introduces 35
linear and non-linear constraint equations on the optimization variables. We
formulate a constrained optimization problem and use nonlinear programming to
discover optimal, yet realizable, bicycle designs.

\section*{Introduction}
%
We introduced a method of optimizing four geometric parameters of a bicycle's
design to minimize the so called Handling Quality Metric
(HQM)~\cite{Moore2016}. Here we expand that method to optimize all X geometric
and inertial parameters of the benchmark parameterimization of the linear
Whipple-Carvallo bicycle model under constraints that guarantee a physically
realizable bicycle. This improves over the prior work by expanding the search
space with many more parameters and the guarntee of realizability.

\section*{Problem Definition}
%
We search for 24 optimization variables that minimize the Handling Quality
Metric defined in~\cite{Hess2012}.

\subsection*{Optimization}
%
The constraint statements above translate into upper and lower bounds on 12 of
the 24 variables $\mathbf{p}^U,\mathbf{p}^L$ and 14 linear and 21 nonlinear
constraint $\mathbf{g}(\mathbf{p})$ among the variables and thus can be
expressed in the standard non-linear programming form.
%
\begin{equation}
  \begin{aligned}
    & \underset{\mathbf{p}}{\text{minimize}} & & max(\textrm{HQM}(\mathbf{p})) \\
    & \text{subject to} & & \\
    & & & \mathbf{p}^L \leq \mathbf{p} \leq \mathbf{p}^U \\
    & & & \mathbf{g}^L \leq \mathbf{g}(\mathbf{p}) \leq \mathbf{g}^U \\
  \end{aligned}
\end{equation}

\subsection*{Bicycle Model Parameters}
%
Our problem formulation relies on a new parameterization that reflects both a
reformulation of and addition to the benchmark parameterization of the linear
Whipple-Carvallo bicycle model~\cite{Meijaard2007}. This parameterization
differs from the benchmark parameterization in that the person and rear frame
are treated as separate but fixed rigid bodies each with their on inertial
parameters and that the inertial parameters of all bodies are expressed in
terms of central principal radii of gyration so that the mass is decoupled from
the inertia. We also introduce two simple dimensional parameters that define
the geometric extents of the person. Table~\ref{tab:parameters} provides the
parameters and the initial values which are derived from the measurements of a
Batuvus Browser Bicycle and the rider, Jason, presented in~\cite{Moore2012}.
This allows the position and orientation of the rider to be specified
independtly of the rear frame but does not introduce any more than the
benchmark's four rigid bodies.

The variables are as follows:
%
\begin{itemize}
  \item wheelbase, trail, steer axis tilt, wheel radii
  \item somersault angle of the person (same direction as bicycle pitch)
  \item masses of the front frame, rear frame, front wheel, and rear wheel
  \item mass center locations of the front frame, rear frame, and rider
  \item principal central radii of gyration of front and rear frames
  \item principal inertia axis pitch angles of the front frame, rear frame, and rider
\end{itemize}


\begin{table}
  \caption{Parameters and their default values.}
  \begin{tabular}{llll}
    \toprule
    Variable & Value & Units & Description \\
    \midrule
    $c$ &  0.068581 & \si{\meter} & Trail \\
    $w$ &  1.1210 & \si{\meter} & Wheelbase \\
    $\lambda$ &  0.39968 & \si{\radian} & Steer axis tilt \\
    $g$ &  9.81 & \si{\meter\per\second\squared} & Acceleration due to gravity \\
    $v$ &  3.0 & \si{\meter\per\second} & Forward speed \\
    Rear Wheel [R] & & \\
    \midrule
    $m_R$ &  3.11 & \si{\kilogram} & mass \\
    $r_R$ &  0.34096 & \si{\meter} & Rear wheel radius \\
    $x_R$ & 0 & \si{\meter} & Rear wheel mass center \\
    $y_R$ & 0 & \si{\meter} & Rear wheel mass center \\
    $z_R$ & -0.34096 & \si{\meter} & Rear wheel mass center \\
    $k_{Raa}$ &  0.17050 & \si{\meter} & Rear wheel central principal radii of gyration \\
    $k_{Rbb}$ &  0.17050 & \si{\meter} & Rear wheel central principal radii of gyration  \\
    $k_{Ryy}$ &  0.22136 & \si{\meter} & Rear wheel central principal radii of gyration \\
    Front Wheel [F] & & \\
    \midrule
    $m_F$ &  2.02 & \si{\kilogram} & Mass \\
    $r_F$ &  0.34353 & \si{\meter} & Radius \\
    $x_F$ &  1.1210 & \si{\meter} & Mass center \\
    $y_F$ & 0.0 & \si{\meter} & Mass center \\
    $z_F$ & -0.34353 & \si{\meter} & Mass center \\
    $k_{Faa}$ &  0.20917 & \si{\meter} & Central principal radius of gyration \\
    $k_{Fbb}$ &  0.20917 & \si{\meter} & Central principal radius of gyration \\
    $k_{Fyy}$ &  0.27179 & \si{\meter} & Central principal radius of gyration \\
    Person [P] & & \\
    \midrule
    $l_P$ &  1.7280 & \si{\meter} & Body length \\
    $w_P$ &  0.48300 & \si{\meter} & Body width \\
    $m_P$ &  83.500 & \si{\kilogram} & Mass \\
    $x_P$ &  0.31577 & \si{\meter} & Mass center \\
    $y_P$ & 0.0 & \si{\meter} & Mass center \\
    $z_P$ & -1.0990 & \si{\meter} & Mass center \\
    $k_{Paa}$ &  0.36797 & \si{\meter} & Central principal radius of gyration \\
    $k_{Pbb}$ &  0.15276 & \si{\meter} & Central principal radius of gyration \\
    $k_{Pyy}$ &  0.36717 & \si{\meter} & Central principal radius of gyration \\
    $\alpha_P$ & 0.18618 & \si{\radian} & Principal axis angle \\
    Front Frame [H] & & \\
    \midrule
    $m_H$ & 3.2200 & \si{\kilogram} & Mass \\
    $x_H$ & 0.86695 & \si{\meter} & Mass center \\
    $y_H$ & 0.0 & \si{\meter} & Mass center \\
    $z_H$ & -0.74824 & \si{\meter} & Mass center \\
    $k_{Haa}$ & 0.29556 & \si{\meter} & Central principal radius of gyration \\
    $k_{Hbb}$ & 0.14493 & \si{\meter} & Central principal radius of gyration \\
    $k_{Hyy}$ & 0.27630 & \si{\meter} & Central principal radius of gyration \\
    $\alpha_H$ & 0.36995 & \si{\radian} & Principal axis angle \\
    Rear Frame [D] & & \\
    \midrule
    $m_D$ &  9.8600 & \si{\kilogram} & Mass \\
    $x_D$ &  0.27595 & \si{\meter} & Mass center \\
    $y_D$ & 0 & \si{\meter} & Mass center \\
    $z_D$ & -0.53784 & \si{\meter} & Mass center \\
    $k_{Daa}$ &  0.28587 & \si{\meter} & Central principal radius of gyration \\
    $k_{Dbb}$ &  0.22079 & \si{\meter} & Central principal radius of gyration \\
    $k_{Dyy}$ &  0.36539 & \si{\meter} & Central principal radius of gyration \\
    $\alpha_D$ &  1.1722 & \si{\radian} & Principal axis angle \\
    \bottomrule
  \end{tabular}
\end{table}

$\mathbf{p}\in\mathbb{R}^{24}$These parameters can be transformed into the benchmark parameterization
r adily, but not vice versa.
We search for the optimal values of a subset of 23 of the principal parameters.

\begin{align}
  \mathbf{p} = &
    [w \quad
     c \quad
     \lambda \quad
     m_D \quad
     x_D \quad
     z_D \quad
     k_{Daa} \quad
     k_{Dbb} \quad
     \alpha_D \quad
     x_P \quad
     z_P \quad
     \alpha_P \quad \\
 &    m_H \quad
     x_H \quad
     z_H \quad
     k_{Haa} \quad
     k_{Hbb} \quad
     k_{Hyy} \quad
     \alpha_H \quad
     r_R \quad
     m_R \quad
     r_F \quad
     m_F]
\end{align}

\subsection*{Bounds and Constraints}
%
The optimal principal parameters are subject to a set of constraints designed
to ensure a physically realizable bicycle is obtained.
We formulate bounds on the variables as well as equality and inequality
constraints among the variables to ensure a physically realizable bicycle.
Below the constraints are defined and grouped by the associated rigid body or
collection thereof:
%
\begin{description}
  \item[Vehicle] The resulting combination of the five rigid bodies.
    \begin{itemize}
      \itemsep0em
      \item The physical extents of the bicycle and rider must exist above the
        ground plane.
      \item Both bicycle and rider are symmetric about the sagittal
        plane.
      \item The mass of each bicycle rigid body is positive, greater than a
        minimum value, and the total mass is below a reasonably lift-able
        amount.
      \item The wheels cannot overlap, i.e. wheelbase > front radius + rear
        radius.
      \item The bicycle cannot topple forward during hard braking or backward
        during hard acceleration, i.e. -1 g < acceleration < 1 g.
    \end{itemize}

  \item[Wheels] Both front and rear wheels have identical constraints.
    \begin{itemize}
      \itemsep0em
      \item Wheel radius must be greater than a minimum value.
      \item Wheels are inertially wheel-like, i.e.  $I_{rot}=mr^2,
        I_{rad}=I_{rot}/2$.
    \end{itemize}

  \item[Frames] Both front and rear frames have identical constraints.
    \begin{itemize}
      \itemsep0em
      \item A pair of principal directions lie in the roll/yaw
        plane to maintain symmetry and one of these axes is within $\pm \pi/4$
        rad of the ground plane.
      \item The mass and inertia of the frames are positive and large enough to
        be constructed from materials in a space frame of specified minimal
        density.
      \item The frames are planar in nature, i.e.
        $k_{lateral}^2=k_{longitudinal}^2+k_{vertical}^2$, where $k$ is the
        radius of gyration.
    \end{itemize}

  \item[Rider] A single rigid body represents the rider.
    \begin{itemize}
      \itemsep0em
      \item Rider mass is that of an average person.
      \item The rider's joint angles are fixed in a nominal configuration
        typical of upright bicycling and the resulting mass distribution is
        derived from standard body segment estimation methods.
      \item One rider principal axis is parallel with the bicycle pitch axis to
        enforce symmetry.
    \end{itemize}

\end{description}


\begin{table}
  \caption{Inequality constraints}
  \begin{tabular}{lll}
    \toprule
    Constraint & Equation & Description \\
    \midrule
    $c_1$    & $\sqrt{I_{Hxx}^2+I_{Hzz}^2} \geq I_{Hyy}$           & Narrow front frame. \\
    $c_2$    & $0 \geq z_P + l_P/2 \cos{\alpha_P}$                 & Person cannot penetrate ground. \\
    $c_3$    & $0 \geq z_P + w_P/2 \sin{\alpha_P}$                 & Person cannot penetrate ground. \\
    $c_4$    & $0 \geq z_P - l_P/2 \cos{\alpha_P}$                 & Person cannot penetrate ground. \\
    $c_5$    & $0 \geq z_P - w_P/2 \sin{\alpha_P}$                 & Person cannot penetrate ground. \\
    $c_6$    & $x_T \geq |z_T|/4$                                  & Maximum acceleration of $1/4g$. \\
    $c_7$    & $w-x_T \geq 3/4|z_T|$                               & Maximum deceleration of $3/4g$. \\
    $c_8$    & $2k_{Hyy} \geq \sqrt{(x_H - w)^2 + (z_H + r_F)^2}$  & Minimal inertial spread. \\
    $c_9$    & $2k_{Dyy} \geq \sqrt{(x_D - 0)^2 + (z_D + r_R)^2}$  & Minimal inertial spread. \\
    $c_{10}$ & $w \geq r_F +r_R$                                   & Non-overlapping wheels. \\
    $c_{11}$ & $25\si{\kg} \geq m_D + m_H + m_R + m_F$             & Maximum bicycle mass. \\
    $c_{12}$ & $-z_D \geq 1.4 k_{Dyy}$                             & Rear frame cannot penetrate ground. \\
    $c_{13}$ & $-z_H \geq 1.4 k_{Hyy}$                             & Front frame cannot penetrate ground. \\
    $c_{14,\ldots,21}$ & $0 \geq s_1,\ldots,s_8$                   & Closed loop stability. \\
    \bottomrule
  \end{tabular}
\end{table}

\begin{table}
  \caption{Equality constraints}
  \begin{tabular}{lll}
    \toprule
    Constraint & Equation & Description \\
    \midrule
    $g_1$ & $I_{Dyy} = \sqrt{I_{Dxx}^2 + I_{Dzz}^2}$ & Rear frame is planar. \\
    $g_2$ & $k_{Ryy} = r_R$  & Rear wheel is a ring \\
    $g_3$ & $k_{Raa} = k_{Ryy}/2$ & Rear wheel is a ring \\
    $g_4$ & $k_{Rbb} = k_{Ryy}/2$ & Rear wheel is a ring \\
    $g_5$ & $k_{Fyy} = r_F$  & Front wheel is a ring \\
    $g_6$ & $k_{Faa} = k_{Fyy}/2$ & Front wheel is a ring \\
    $g_7$ & $k_{Fbb} = k_{Fyy}/2$ & Front wheel is a ring \\
    \bottomrule
  \end{tabular}
\end{table}

\begin{table}
  \caption{Parameter Bounds}
  \begin{tabular}{ccccc}
    \toprule
    Min & & Parmeter & & Max\\
    \midrule
    $-\infty$ & $\leq$ & $w       $  &  $\leq$  &  $\infty$ \\
    $-\infty$ & $\leq$ & $c       $  &  $\leq$  &  $\infty$ \\
    $-\pi/2 $ & $\leq$ & $\lambda $  &  $\leq$  &  $\pi/2$ \\
    $1.0    $ & $\leq$ & $m_D     $  &  $\leq$  &  $\infty$ \\
    $-\infty$ & $\leq$ & $x_D     $  &  $\leq$  &  $\infty$ \\
    $-\infty$ & $\leq$ & $z_D     $  &  $\leq$  &  $0.0$    \\
    $0.0    $ & $\leq$ & $k_{Daa} $  &  $\leq$  &  $\infty$ \\
    $0.0    $ & $\leq$ & $k_{Dbb} $  &  $\leq$  &  $\infty$ \\
    $-\pi/2 $ & $\leq$ & $\alpha_D$  &  $\leq$  &  $\pi/2$  \\
    $-\infty$ & $\leq$ & $x_P     $  &  $\leq$  &  $\infty$ \\
    $-\infty$ & $\leq$ & $z_P     $  &  $\leq$  &  $0.0$    \\
    $-\pi/2 $ & $\leq$ & $\alpha_P$  &  $\leq$  &  $\pi/2$ \\
    $0.25   $ & $\leq$ & $m_H     $  &  $\leq$  &  $\infty$ \\
    $-\infty$ & $\leq$ & $x_H     $  &  $\leq$  &  $\infty$ \\
    $-\infty$ & $\leq$ & $z_H     $  &  $\leq$  &  $0.0$ \\
    $0.0    $ & $\leq$ & $k_{Haa} $  &  $\leq$  &  $\infty$ \\
    $0.0    $ & $\leq$ & $k_{Hbb} $  &  $\leq$  &  $\infty$ \\
    $0.0    $ & $\leq$ & $k_{Hyy} $  &  $\leq$  &  $\infty$ \\
    $-\pi/2 $ & $\leq$ & $\alpha_H$  &  $\leq$  &  $\pi/2$ \\
    $0.127  $ & $\leq$ & $r_R     $  &  $\leq$  &  $\infty$ \\
    $1.0    $ & $\leq$ & $m_R     $  &  $\leq$  &  $\infty$ \\
    $0.127  $ & $\leq$ & $r_F     $  &  $\leq$  &  $\infty$ \\
    $1.0    $ & $\leq$ & $m_F     $  &  $\leq$  &  $\infty$ \\
    \bottomrule
  \end{tabular}
\end{table}

\section*{Rigid Body Definitions}

\subsection*{Rear Frame [D]}
%
There are five free parameters associated with the rear frame's inertial
characteristics:

\begin{description}
  \item[$m_D$] Mass of the rear frame.
  \item[$x_D$] Location along the $X$ axis of the rear frame center of mass.
  \item[$z_D$] Location along the $Z$ axis of the rear frame center of mass.
  \item[$k_{Daa}$] Maximum principal central radius of gyration in the $XZ$
    plane.
  \item[$k_{Dbb}$] Minimum principal central radius of gyration in the $XZ$
    plane.
  \item[$\alpha_D$] Angle between the $X$ axis and the maximum principal axis
    that lies in the $XZ$ plane.
\end{description}

Several constraints are set for the rear frame. We constrain the rear frame to
be planar, $g_1$ and symmetric with respect to the $XZ$ plane, i.e. $I_{Dxy} =
I_{Dyz} = 0$. We prevent the rear frame from penetrating the ground by limiting
the inertial spread with respect to its mass center, $c_{12}$, but also set a
minimum inertial spread to ensure a frame can span from the rear wheel to the
mass center of the rear frame, $c_9$. The spread factor in $c_{12}$ of 1.4 is
based on the ratio of geometrical spread of a typical bicycle frame and its
radius of gyration. Several parameters are bounded. We require the rear frame
mass to be positive, the center of mass not penetrate the ground, and we allow
for any angular orientation of the $XZ$ principal directions but limit the
angle to $-\frac{\pi}{2} \leq \alpha_D \leq \frac{\pi}{2}$.

\subsection{Person [P]}

We assume that the person's joint configurations are such that they are in a
nominal configuration for pedaling, i.e. an average normal everyday riding
position on a typical bicycle. The person is assumed to be symmetric about the
$XZ$ plane. We allow the rider to be rotated about the $Y$ axis and positioned
anywhere within the plane of symmetry above the ground. Three free parameters
are associated with these assumptions:

\begin{description}
  \item[$x_P$] Location along the $X$ axis of the person's center of mass.
  \item[$z_P$] Location along the $Z$ axis of the person's center of mass.
  \item[$\alpha_P$] Angle between the $X$ axis and the primary principal
    inertia axis of the person.
\end{description}

There are four fixed parameters that enforce this assumption:

\begin{description}
  \item[$m_P$] The mass of the person which is fixed at $XX \si{kg}$.
  \item[$k_{Pxx}$] Central radius of gyration with respect to the $X$ axis
    where $I_{Pxx} = m_P k_{Pxx}^2$.
  \item[$k_{Pyy}$] Central radius of gyration with respect to the $Y$ axis
    where $I_{Pyy} = m_P k_{Pyy}^2=$.
  \item[$k_{Pzz}$] Central radius of gyration with respect to the $Z$ axis
    where $I_{Pzz} = m_P k_{Pzz}^2=$.
  \item[$I_{Bxy}, I_{Byz}$] Products of inertia are zero.
  \item[$y_P$] Is zero.
\end{description}

To prevent the rider from being positioned and oriented such that their body is
not penetrating the the ground we introduce two dimensions that define a cross
whose apex is at the center of mass of the person and the cross axes are
parallel to the principal axes in the $XZ$ plane. $l_P / 2$ is the distance
along the principal axis to the tip of the toes and $w_P$ is the distance along
the second principal axes to the tip of the hands. Constraints $c_2,\ldots,c_5$.

\subsection{Front Frame}

The front frame is symmetric about the $XZ$ plane so $I_{Hxy}, I_{Hyz} = 0$.
There are seven free parameters:

\begin{description}
  \item[$m_H$] The mass of the rear frame.
  \item[$x_B$] Location along the $X$ axis of the rear frame's center of mass.
  \item[$z_H$] Location along the $Z$ axis of the rear frame's center of mass.
  \item[$k_{Haa}$] Central radius of gyration with respect to the $X$ axis
    where $k_{Hxx}^2=I_{Hxx} / m_H$.
  \item[$k_{Hyy}$] Central radius of gyration with respect to the $X$ axis
    where $k_{Hyy}^2=I_{Hyy} / m_H$.
  \item[$k_{Hbb}$] Central radius of gyration with respect to the $X$ axis
    where $k_{Hzz}^2=I_{Hzz} / m_H$.
  \item[$\alpha_H$] Angle between the $X$ axis and the primary principal
    inertia axis.
\end{description}

We allow for any angular orientation of the $XZ$ principal directions but limit
the angle to $-\frac{\pi}{2} \leq \alpha_H \leq \frac{\pi}{2}$.  We prevent the
rear frame from penetrating the ground by limiting the inertial spread with
respect to its mass center, $c_{13}$, but also set a minimum inertial spread to
ensure a frame can span from the rear wheel to the mass center of the rear
frame, $c_8$. The spread factor in $c_{13}$ of 1.4 is based on the ratio of
geometrical spread of a typical bicycle frame and its radius of gyration. The
front frame is not planar but is narrow with respect to the $XZ$ plane, which
is enforced by constraint $c_1$.

\subsection{Front [F] and Rear [R] Wheels}

We enforce the assumption that both wheels have moments of inertia of that of a
simple ring, $g_2 \ldots g_7$ and that mass and radius should be greater than
some minimal size based on small bicycle wheel.

\subsection{Total Bike [T]}

\subsubsection{Parameters}

There are three remaing parameters associated with the fundamental geometry of
the bicycle: wheelbase $w$, trail $c$, and steer axis tilt $\lambda$ all the
same as defined in~\cite{Meijaard2007}.

\subsubsection{Bounds}

The trail can take on any real value as well as the steer axis tilt except the
bound the angle to a minimal range:

\begin{align}
  -\frac{\pi}{2} \leq \lambda \leq \frac{\pi}{2}
\end{align}


\subsubsection{Constraints}

We introduce a constraint $c_{10}$ that prevents the wheels from physically
overlapping and require that the bicycle be liftable by an average person,
$c_{11}$. Finally, we require that the bicycle not topple forward during hard
breaking or backward during hard acceleration.

\begin{align}
  -\frac{3g}{4} < \textrm{acceleration} < \frac{g}{4}
\end{align}

This translates to two constraints, $c_6,c_7$ that bound the total center of
mass $(x_T,z_T)$ in a triangle in the $XZ$ plane.

Lastly, we constrain the eight closed loop eigenvalues associated with the
controller in ~\cite{Hess2012} to be stable, i.e. have negative real parts.
These are expressed in constraints $c_{14},\ldots,c_{21}$. Closed loop
stability is required for the HQM to provide a meaningful result.

\subsection{Summary}

The above definitions represent a constrained optimization in 23 free
parameters with 17 constraints. This bounds the space of bicycle designs to be
physically realizable and excludes bicycles that don't have the inherent
characteristics of a bicycle.

\begin{align}
  \mathbf{p} =
  \begin{bmatrix}
    % primary geometry
    w \\
    c \\
    \lambda
    % front frame
    m_D \\
    x_D \\
    z_D \\
    k_{Dxx} \\
    k_{Dzz} \\
    \alpha_D \\
    % person
    x_P \\
    z_P \\
    \alpha_P \\
    % front frame
    m_H \\
    x_H \\
    z_H \\
    k_{Hxx} \\
    k_{Hyy} \\
    k_{Hzz} \\
    I_{Hxz} \\
    \alpha_H \\
    % rear wheel
    r_R \\
    m_R \\
    % front wheel
    r_F \\
    m_F \\
  \end{bmatrix}
\end{align}





\bibliographystyle{acm}
\bibliography{BMD2019}


\section*{Solving}
\section*{Results}
\section*{Discussion and Conclusion}

\end{document}
