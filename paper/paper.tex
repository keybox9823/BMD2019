\documentclass{bmd2019a}

\usepackage{siunitx}
\usepackage{amsmath}
\usepackage{booktabs}

\begin{document}

\begin{flushleft}
{\fontsize{16pt}{20pt}\selectfont%
  Expanded Optimization for Discovering Optimal Lateral Handling Bicycles}
\end{flushleft}

%%%%%%%%%%%%%%%% authors %%%%%%%%%%%%%%%
\begin{flushleft}
  {\fontsize{12}{14}{Jason K. Moore and Mont Hubbard}\\}
  \textit{Mechanical and Aerospace Engineering\\
          University of California, Davis\\
          One Shields Avenue, Davis, CA, USA 95817\\
          e-mail: jkm@ucdavis.edu, mhubbard@ucdavis.edu}
\end{flushleft}

% - Abstract
% - Introduction
% - Parameterization
% - Bounds and Constraints
% - Free Parameters
% - Optimization
% - Solutions
%   - Parameters
%   - Eigenvalues
%   - HQM
% - Discussion and Conclusion

\section*{Abstract}
%
Previously, we introduced a method of optimizing four primary geometric
parameters of a bicycle's design to minimize its Handling Quality Metric. Here
we expand that method to optimize 23 geometric and inertial parameters in the
benchmark parameterimization of the linear Whipple-Carvallo bicycle model. To
ensure physically realizable bicycle designs we include 7 equality constraints,
21 inequality constraints, and maximal and minamal bounds on each free
opmization parameter. This improves over the prior work by expanding the search
space with many more parameters and the guarntee of realizability. We close by
presenting five bicycle designs that have optimal later handling qualities. The
bicycles are not generally self-stable and have some ununusal characteristics.

\section{Introduction}
%
Physical design features of ground vehicles can affect their lateral handling
qualities. Geometry, mass, and mass distribution of the vehicle's components as
well as tire characteristics are primary contributors to poor and good handling
due to their important influence on the vehicle's dynamics. In past work, we
have presented a theoretical and computational framework for assessing the
lateral task-independent handling qualities of simplified single track vehicle
designs~\cite{Hess2012,Moore2012}. In subsequent work, we showed that
minimizing our proposed handling quality metric (HQM) can produce theoretically
optimal handling designs when only four geometric parameters are explored as
the optimization variables~\cite{Moore2016}. The present work's goal is to
expand this optimization problem to all of the geometry, mass, and inertial
parameters present in the linear Whipple-Carvallo bicycle
model~\cite{Meijaard2007}.  This broadens the search space considerably but we
constrain it so that only realizable optimal bicycle designs are discovered. To
do so, we formulate a constrained optimization problem and use derivative-free
optimization to discover optimal, yet realizable, bicycle designs.

\section{Bicycle Model Parameterization}
%
Our problem formulation relies on a new bicycle model parameterization that
reflects both a reformulation of and addition to the benchmark parameterization
of the linear Whipple-Carvallo bicycle model~\cite{Meijaard2007}. We call this
the ``principal parameterization'' as opposed to the ``benchmark
parameterization'' in~\cite{Meijaard2007}. This parameterization differs from
the benchmark parameterization in three ways. Firstly, the person and rear
frame are treated as separate rigid bodies each with their on inertial
parameters. Secondly, we express the inertial parameters of each rigid body in
terms of central principal radii of gyration to decouple the mass from the
inertia terms.  Lastly, we introduce two simple dimensional parameters that
define the geometric extents of the person which are used to constrain the
location of the person's body. Table~\ref{tab:parameters} provides the
parameters and the reference values which are derived from the measurements of
a Batavus Browser Bicycle and the rider ``Jason'' presented
in~\cite{Moore2012}. This parameterization can be transformed into the
benchmark parameterization readily, but not vice versa.
%
\begin{figure}
  \centering
  \includegraphics[width=3in]{../figures/reference-geometry.png}
  \caption{Depiction of the principal parameters of the Batavus Browser
    with rider ``Jason''. The solid black lines represent the essential bicycle
    geometry. The dotted black line represents the steer axis. The inertial
    properties of the five rigid bodies: front wheel (orange), rear wheel
    (purple), rear frame (blue), front frame (green), and person (red) are
    shown with the mass center and the extents of the centroidal principal
    radii of gyration of each rigid body. The primary principal angles,
    $\alpha_{D,P,H}$, are defined as the angle about the $y$ axis from $x$ to
    the maximum principal axes in the $XZ$ plane.}
\end{figure}
%
\begin{table}
  \caption{Full set of 47 principal parameters and their default values derived
    from the measurements in~\cite{Moore2012} of the Batavus Browser bicycle
    and rider ``Jason''. Note that the benchmark parameters are also used in
    the paper and are defined in~\cite{Meijaard2007}.}
  \label{tab:parameters}
  \small
  \centering
  \begin{tabular}{llll}
    \toprule
    Variable & Value & Units & Description \\
    \midrule
    % 5
    $c$       &  0.068581           & \si{\meter} & Trail \\
    $w$       &  1.1210             & \si{\meter} & Wheelbase \\
    $\lambda$ & 0.39968             & \si{\radian} & Steer axis tilt \\
    $g$       &  9.81               & \si{\meter\per\second\squared} & Acceleration due to gravity \\
    $v$       &  3.0, 5.0, 7.0, 9.0 & \si{\meter\per\second} & Forward speed \\
    Rear Wheel [R] & & \\
    \midrule
    % 8
    $m_R$     & 3.11 & \si{\kilogram}  & Mass \\
    $r_R$     & 0.34096 & \si{\meter}  & Radius \\
    $x_R$     & 0.0 & \si{\meter}      & $X$ mass center coordinate \\
    $y_R$     & 0.0 & \si{\meter}      & $Y$ mass center coordinate \\
    $z_R$     & -0.34096 & \si{\meter} & $Z$ mass center coordinate \\
    $k_{Raa}$ & 0.17050 & \si{\meter}  & Central principal radii of gyration about $A_R$ \\
    $k_{Rbb}$ & 0.17050 & \si{\meter}  & Central principal radii of gyration about $B_R$ \\
    $k_{Ryy}$ & 0.22136 & \si{\meter}  & Central principal radii of gyration about $Y$ \\
    Front Wheel [F] & & \\
    \midrule
    % 8
    $m_F$     & 2.02 & \si{\kilogram} & Mass \\
    $r_F$     & 0.34353 & \si{\meter} & Radius \\
    $x_F$     & 1.1210 & \si{\meter} & $X$ mass center coordinate \\
    $y_F$     & 0.0 & \si{\meter} & $Y$ mass center coordinate \\
    $z_F$     & -0.34353 & \si{\meter} & $Z$ mass center coordinate \\
    $k_{Faa}$ &  0.20917 & \si{\meter} & Central principal radii of gyration about $A_F$ \\
    $k_{Fbb}$ &  0.20917 & \si{\meter} & Central principal radii of gyration about $B_F$ \\
    $k_{Fyy}$ &  0.27179 & \si{\meter} & Central principal radii of gyration about $Y$ \\
    Person [P] & & \\
    \midrule
    % 10
    $l_P$      &  1.728   & \si{\meter} & Body length \\
    $w_P$      &  0.483   & \si{\meter} & Body width \\
    $m_P$      &  83.5    & \si{\kilogram} & Mass \\
    $x_P$      &  0.31577 & \si{\meter} & $X$ mass center coordinate \\
    $y_P$      & 0.0      & \si{\meter} & $Y$ mass center coordinate \\
    $z_P$      & -1.0990  & \si{\meter} & $Z$ mass center coordinate \\
    $k_{Paa}$  &  0.36797 & \si{\meter} & Central principal radii of gyration about $A_P$ \\
    $k_{Pbb}$  &  0.15276 & \si{\meter} & Central principal radii of gyration about $B_P$ \\
    $k_{Pyy}$  &  0.36717 & \si{\meter} & Central principal radii of gyration about $Y$ \\
    $\alpha_P$ & 0.18618 & \si{\radian} & Angle about $Y$ between $A_P$ and $X$ \\
    Front Frame [H] & & \\
    \midrule
    % 8
    $m_H$ & 3.2200 & \si{\kilogram}     & Mass \\
    $x_H$ & 0.86695 & \si{\meter}       & $X$ mass center coordinate \\
    $y_H$ & 0.0 & \si{\meter}           & $Y$ mass center coordinate \\
    $z_H$ & -0.74824 & \si{\meter}      & $Z$ mass center coordinate \\
    $k_{Haa}$ & 0.29556 & \si{\meter}   & Central principal radii of gyration about $A_H$ \\
    $k_{Hbb}$ & 0.14493 & \si{\meter}   & Central principal radii of gyration about $B_H$ \\
    $k_{Hyy}$ & 0.27630 & \si{\meter}   & Central principal radii of gyration about $Y$ \\
    $\alpha_H$ & 0.36995 & \si{\radian} & Angle about $Y$ between $A_H$ and $X$ \\
    Rear Frame [D] & & \\
    \midrule
    % 8
    $m_D$ &  9.8600 & \si{\kilogram}    & Mass \\
    $x_D$ &  0.27595 & \si{\meter}      & $X$ mass center coordinate \\
    $y_D$ & 0 & \si{\meter}             & $Y$ mass center coordinate \\
    $z_D$ & -0.53784 & \si{\meter}      & $Z$ mass center coordinate \\
    $k_{Daa}$ &  0.28587 & \si{\meter}  & Central principal radii of gyration about $A_D$ \\
    $k_{Dbb}$ &  0.22079 & \si{\meter}  & Central principal radii of gyration about $B_D$ \\
    $k_{Dyy}$ &  0.36539 & \si{\meter}  & Central principal radii of gyration about $Y$ \\
    $\alpha_D$ &  1.1722 & \si{\radian} & Angle about $Y$ between $A_D$ and $X$ \\
    \bottomrule
  \end{tabular}
\end{table}

\section{Bounds and Constraints}
%
The optimal principal parameters are subject to a set of constraints designed
to ensure that a physically realizable bicycle is obtained from the
optimization procedure. These constraints are made up of bounds on the free
parameters and both equality and inequality constraints among the parameters.
Below the basic constraint concepts presented and grouped by the associated
rigid body or collection thereof:
%
\begin{description}
  \item[Total $T$] The combination of the five rigid bodies.
    \begin{itemize}
      \item The likely physical extents of the rigid bodies must exist above
        the ground plane.
      \item Both bicycle and rider are symmetric about the rider's sagittal
        plane.
      \item The total mass is below a reasonably human lift-able amount.
      \item The wheels cannot overlap.
      \item The bicycle cannot topple forward during hard braking or backward
        during hard acceleration.
      \item The closed-loop path tracking controlled system~\cite{Hess2012}
        must be stable. This is required to obtain a valid HQM value.
    \end{itemize}

  \item[Rider $P$] A single rigid body represents the rider.
    \begin{itemize}
      \item Rider mass is that of an average person.
      \item The rider's joint angles are fixed in a nominal configuration
        typical of upright bicycling and the resulting mass distribution is
        derived from standard body segment estimation methods.
      \item The rider cannot penetrate the ground.
    \end{itemize}

  \item[Frames $H$, $D$] Front frame (handlebar + fork) and rear frame
    \begin{itemize}
      \item The rear frame is planar in nature and the front frame's moments of
        inertia are consistently dependent.
      \item The mass and inertia of the frames are positive and large enough to
        be constructed from a steel space frame.
    \end{itemize}

  \item[Wheels $F$, $R$] Both front and rear wheels have identical constraints.
    \begin{itemize}
      \item Wheel radius and mass must be positive and be greater than a
        minimum value.
      \item Wheels are inertially wheel-like, i.e. symmetric about each plane
        and most of the mass is at the rim.
    \end{itemize}

\end{description}

These bounds, equality, and inequality constraints are presented mathematically
in tables \ref{tab:bounds}, \ref{tab:equality-constraints},
\ref{tab:inequality-constraints} respectively and explained in more detail in
the following sections.
%
\begin{table}
  \caption{Parameter lower and upper bounds.}
  \label{tab:bounds}
  \centering
  \begin{tabular}{ccccc}
    \toprule
    Min & & Parameter & & Max\\
    \midrule
    $-\infty$ & $\leq$ & $w       $  &  $\leq$  &  $\infty$ \\
    $-\infty$ & $\leq$ & $c       $  &  $\leq$  &  $\infty$ \\
    $-\pi/2 $ & $\leq$ & $\lambda $  &  $\leq$  &  $\pi/2$ \\
    $1.0\si{\kilogram}$ & $\leq$ & $m_D     $  &  $\leq$  &  $\infty$ \\
    $-\infty$ & $\leq$ & $x_D     $  &  $\leq$  &  $\infty$ \\
    $-\infty$ & $\leq$ & $z_D     $  &  $\leq$  &  $0.0$    \\
    $0.0    $ & $\leq$ & $k_{Daa} $  &  $\leq$  &  $\infty$ \\
    $0.0    $ & $\leq$ & $k_{Dbb} $  &  $\leq$  &  $\infty$ \\
    $-\pi/2 $ & $\leq$ & $\alpha_D$  &  $\leq$  &  $\pi/2$  \\
    $-\infty$ & $\leq$ & $x_P     $  &  $\leq$  &  $\infty$ \\
    $-\infty$ & $\leq$ & $z_P     $  &  $\leq$  &  $0.0$    \\
    $-\pi/2 $ & $\leq$ & $\alpha_P$  &  $\leq$  &  $\pi/2$ \\
    $0.25   $ & $\leq$ & $m_H     $  &  $\leq$  &  $\infty$ \\
    $-\infty$ & $\leq$ & $x_H     $  &  $\leq$  &  $\infty$ \\
    $-\infty$ & $\leq$ & $z_H     $  &  $\leq$  &  $0.0$ \\
    $0.0    $ & $\leq$ & $k_{Haa} $  &  $\leq$  &  $\infty$ \\
    $0.0    $ & $\leq$ & $k_{Hbb} $  &  $\leq$  &  $\infty$ \\
    $0.0    $ & $\leq$ & $k_{Hyy} $  &  $\leq$  &  $\infty$ \\
    $-\pi/2 $ & $\leq$ & $\alpha_H$  &  $\leq$  &  $\pi/2$ \\
    $0.127\si{\meter}$ & $\leq$ & $r_R     $  &  $\leq$  &  $\infty$ \\
    $1.0\si{\kilogram}$ & $\leq$ & $m_R     $  &  $\leq$  &  $\infty$ \\
    $0.127\si{\meter}$ & $\leq$ & $r_F     $  &  $\leq$  &  $\infty$ \\
    $1.0\si{\kilogram}$ & $\leq$ & $m_F     $  &  $\leq$  &  $\infty$ \\
    \bottomrule
  \end{tabular}
\end{table}
%
\begin{table}
  \caption{Equality constraints.}
  \label{tab:equality-constraints}
  \centering
  \begin{tabular}{lll}
    \toprule
    Constraint & Equation & Description \\
    \midrule
    $g_1$ & $I_{Dyy} = \sqrt{I_{Dxx}^2 + I_{Dzz}^2}$ & Rear frame is planar. \\
    $g_2$ & $k_{Ryy} = r_R$  & Rear wheel is a ring \\
    $g_3$ & $k_{Raa} = k_{Ryy}/2$ & Rear wheel is a ring \\
    $g_4$ & $k_{Rbb} = k_{Ryy}/2$ & Rear wheel is a ring \\
    $g_5$ & $k_{Fyy} = r_F$  & Front wheel is a ring \\
    $g_6$ & $k_{Faa} = k_{Fyy}/2$ & Front wheel is a ring \\
    $g_7$ & $k_{Fbb} = k_{Fyy}/2$ & Front wheel is a ring \\
    \bottomrule
  \end{tabular}
\end{table}
%
\begin{table}
  \caption{Inequality constraints.}
  \label{tab:inequality-constraints}
  \centering
  \begin{tabular}{lll}
    \toprule
    Constraint & Equation & Description \\
    \midrule
    $c_1$    & $\sqrt{I_{Hxx}^2+I_{Hzz}^2} \geq I_{Hyy}$           & Consistent moments of inertia. \\
    $c_2$    & $0 \geq z_P + l_P/2 \cos{\alpha_P}$                 & Person cannot penetrate ground. \\
    $c_3$    & $0 \geq z_P + w_P/2 \sin{\alpha_P}$                 & Person cannot penetrate ground. \\
    $c_4$    & $0 \geq z_P - l_P/2 \cos{\alpha_P}$                 & Person cannot penetrate ground. \\
    $c_5$    & $0 \geq z_P - w_P/2 \sin{\alpha_P}$                 & Person cannot penetrate ground. \\
    $c_6$    & $x_T \geq |z_T|/4$                                  & Maximum acceleration of $1/4g$. \\
    $c_7$    & $w-x_T \geq 3/4|z_T|$                               & Maximum deceleration of $3/4g$. \\
    $c_8$    & $2k_{Hyy} \geq \sqrt{(x_H - w)^2 + (z_H + r_F)^2}$  & Minimal inertial spread. \\
    $c_9$    & $2k_{Dyy} \geq \sqrt{(x_D - 0)^2 + (z_D + r_R)^2}$  & Minimal inertial spread. \\
    $c_{10}$ & $w \geq r_F +r_R$                                   & Non-overlapping wheels. \\
    $c_{11}$ & $25\si{\kg} \geq m_D + m_H + m_R + m_F$             & Maximum bicycle mass. \\
    $c_{12}$ & $-z_D \geq 1.4 k_{Dyy}$                             & Rear frame cannot penetrate ground. \\
    $c_{13}$ & $-z_H \geq 1.4 k_{Hyy}$                             & Front frame cannot penetrate ground. \\
    $c_{14,\ldots,21}$ & $0 \geq s_1,\ldots,s_8$                   & Closed loop stability. \\
    \bottomrule
  \end{tabular}
\end{table}

\subsection{Person [P]}
%
We assume that the person's joint configurations are such that they are in a
nominal configuration for pedaling, i.e. an average normal everyday riding
position on a typical bicycle. We retain the same configuration as they were
seated on the Batavus Browser bicycle. The person is assumed to be symmetric
about the $XZ$ plane. We allow the rider to be rotated about the $Y$ axis and
positioned anywhere within the plane of symmetry above the ground.

To prevent the rider from being positioned and oriented such that their body is
not penetrating the ground we introduce two dimensions that define a cross
whose apex is at the center of mass of the person and the cross axes are
parallel to the principal axes in the $XZ$ plane. $l_P / 2$ is the distance
along the principal axis to the tip of the toes and $w_P / 2$ is the distance
along the second principal axes to the tip of the hands. The constraints
$c_2,\ldots,c_5$ are derived from these rules.

\subsection{Front Frame [H]}
%
The front frame is symmetric about the $XZ$ plane so $I_{Hxy}, I_{Hyz} = 0$. We
allow for any angular orientation of the $XZ$ principal directions but limit
the angle to $-\frac{\pi}{2} \leq \alpha_H \leq \frac{\pi}{2}$. We prevent the
rear frame from penetrating the ground by limiting the inertial spread with
respect to its mass center, $c_{13}$, but also set a minimum inertial spread to
ensure a frame can span from the rear wheel to the mass center of the rear
frame, $c_8$. The spread factor in $c_{13}$ of 1.4 is based on the ratio of
geometrical spread of a typical bicycle frame and its radius of gyration. The
front frame is not planar but is narrow with respect to the $XZ$ plane, which
is enforced by constraint $c_1$.

\subsection{Rear Frame [D]}
%
Several constraints are set for the rear frame. We constrain the rear frame to
be planar, $g_1$, and symmetric with respect to the $XZ$ plane. We prevent the
rear frame from penetrating the ground by limiting the inertial spread with
respect to its mass center, $c_{12}$, but also set a minimum inertial spread to
ensure a frame can span from the rear wheel to the mass center of the rear
frame, $c_9$. The spread factor in $c_{12}$ of 1.4 is based on the ratio of
geometrical spread of a typical bicycle frame and its radius of gyration.
Several parameters are bounded. We require the rear frame mass to be positive,
the center of mass not penetrate the ground, and we allow for any angular
orientation of the $XZ$ principal directions but limit the angle to
$-\frac{\pi}{2} \leq \alpha_D \leq \frac{\pi}{2}$ as angle beyond that are
redundant.

\subsection{Front [F] and Rear [R] Wheels}
%
We enforce the assumption that both wheels have moments of inertia of that of a
simple ring, $g_2 \ldots g_7$ and that mass and radius should be greater than a
minimal size based on small purchasable spoked wheel with tire.

\subsection{Total Bike [T]}
%
The trail and wheelbase can take on any real values. The steer axis tilt is
limited to $\pm90$ degrees. We introduce a constraint $c_{10}$ that prevents
the wheels from physically overlapping and require that the bicycle be
lift-able by an average person, $c_{11}$. Finally, we require that the bicycle
not topple forward during hard breaking or backward during hard acceleration
with:

\begin{align}
  -\frac{3g}{4} < \textrm{acceleration} < \frac{g}{4}\textrm{.}
\end{align}

This translates to two constraints, $c_6,c_7$ that bound the total center of
mass $(x_T,z_T)$ in a triangle in the $XZ$ plane. Lastly, we constrain the
eight closed loop eigenvalues associated with the controller in~\cite{Hess2012}
to be stable, i.e. have negative real parts. These are expressed in
constraints $c_{14},\ldots,c_{21}$. Closed loop stability is required for the
HQM to provide a meaningful result.

\section{Optimization}
%
The above constraints leave 23 of the 47 parameters free for optimizing which
we collect in the vector $\mathbf{p}\in\mathbb{R}^{23}$ and define as:
%
\begin{align}
  \begin{split}
    \mathbf{p} = [ &
       w        \quad
       c        \quad
       \lambda  \quad
       m_D      \quad
       x_D      \quad
       z_D      \quad
       k_{Daa}  \quad
       k_{Dbb}  \quad
       \alpha_D \quad
       x_P      \quad
       z_P      \quad
       \alpha_P \\
     & m_H      \quad
       x_H      \quad
       z_H      \quad
       k_{Haa}  \quad
       k_{Hbb}  \quad
       k_{Hyy}  \quad
       \alpha_H \quad
       r_R      \quad
       m_R      \quad
       r_F      \quad
       m_F]
  \end{split}
\end{align}

Our objective in the optimization is to minimize the peak HQM value subject to
the bounds, $\mathbf{p}^L,\mathbf{p}^U$, and the constraints
$\mathbf{g}(\mathbf{p}),\mathbf{c}(\mathbf{p})$ . Given a set of bicycle model
parameter values we generate a bandwidth limited human-like controller using
the methods in ~\cite{Moore2012}. Once the closed loop stable controller is
constructed, the HQM can be computed as per the definition in \cite{Hess2012}
and the scaler peak value returned as the objective $J$. This problem is
presented as a non-linear programming problem in the following equation.
%
\begin{equation}
  \begin{aligned}
    & \underset{\mathbf{p}}{\text{minimize}} & & J(\mathbf{p})=max(\textrm{HQM}(\mathbf{p})) \\
    & \text{subject to} & & \\
    & & & \mathbf{g}(\mathbf{p}) \leq \mathbf{0} \\
    & & & \mathbf{c}(\mathbf{p}) = \mathbf{0} \\
    & & & \mathbf{p}^L \leq \mathbf{p} \leq \mathbf{p}^U
  \end{aligned}
\end{equation}

We make use of the derivative-free optimizer CMA-ES~\cite{Hansen1996} to find
solutions to this problem. The optimization supports parameter bounds and
equality constraints but does not support inequality constraints. To get around
this limitation we move the inequality constraints into the objective function
and penalize the objective if the constraints are violated with the following
rules:
%
\begin{align}
  J(\mathbf{p}) =
  \begin{cases}
    max(\textrm{HQM}(\mathbf{p})) & \textrm{if} \quad all(\mathbf{g}(\mathbf{p})) \leq 0 \\
    30 + ||\mathbf{g}_{+}(\mathbf{p})||/10 & \textrm{if} \quad
      any(\mathbf{g}(\mathbf{p})) > 0 \textrm{ and } ||\mathbf{g}_{+}(\mathbf{p})|| < 30 \\
    ||\mathbf{g}_{+}(\mathbf{p})|| & \textrm{if} \quad
      any(\mathbf{g}(\mathbf{p})) > 0 \textrm{ and } ||\mathbf{g}_{+}(\mathbf{p})|| \geq 30
  \end{cases}
\end{align}
where $||\mathbf{g}_{+}||$ is the norm of the vector of positive elements of
$\mathbf{g}$ and $any,all$ are ``any elements of'' and ``all elements of'',
respectively.

This creates a discontinuous objective function but in practice the CMA-ES
algorithm is able to move into the parameter space where all the constraints
are satisfied and find a (local) minima. For our purposes, this sufficiently
finds parameter values that produce optimally handling bicycle designs.

\section{Results}
%
We discover four bicycles for different design speeds (3, 5, 7, and
9~\si{\meter\per\second}) that have an optimally low HQM, see
Table~\ref{tab:hqm}, and satisfy all constraints and parameter bounds. The
optimal parameter values for these four bicycles are presented in
Table~\ref{tab:optimal-values}. We belive these bicycles to be reasoably
physically realizable. The pictorial representation of the bicycles are
presented in Figure~\ref{fig:optimal-geometries}.
%
\begin{table}
  \caption{Peak HQM values for the reference bicycle and the optimal bicycles
    at each design speed.}
  \label{tab:hqm}
  \centering
  \begin{tabular}{llll}
    \toprule
    Speed [m/s] & Reference Peak HQM & Optimal Peak HQM & Percent Improvement \\
    \midrule
    3 & 13.075 & 2.012 & 85\% \\
    5 & 4.521  & 0.012 & 100\% \\
    7 & 3.043  & 0.022 & 99\% \\
    9 & 2.338  & 0.839 & 64\% \\
    \bottomrule
  \end{tabular}
\end{table}
%
\begin{table}
  \caption{Optimal principal parameter values for each design speed. Values
    with an asterisk are at a bound.}
  \label{tab:optimal-values}
  \centering
  \begin{tabular}{lSSSS}
    \toprule
    $v$ & 3~\si{\meter\per\second} & 5~\si{\meter\per\second} & 7~\si{\meter\per\second} & 9~\si{\meter\per\second} \\
    \midrule
    $c$       & 0.688  & -0.005 & 0.001  & -0.484 \\
    $w$       & 2.866  & 0.847  & 1.157  & 5.557  \\
    $\lambda$ & -0.213 & 0.271  & -0.028 & 0.239  \\
    \midrule
    $m_D$ & 3.22   &  11.53 & 10.32  & 7.02 \\
    $x_D$ & 0.958  &  0.287 &  0.427 & 1.020 \\
    $z_D$ & -2.605 & -2.719 & -2.976 & -4.065 \\
    $\alpha_D$ & 0.663  & 0.915  & 1.123  & 1.077 \\
    % Should aa be larger than bb?
    $k_{Daa}$  & 1.449  & 1.218  &  1.324 & 0.861 \\
    $k_{Dbb}$  & 0.724  & 1.387  &  1.290 & 2.014 \\
    $k_{Dyy}$  & 1.471  & 1.559  &  1.555 & 2.031 \\
    \midrule
    $m_H$  & 0.25*  & 0.25* &    0.49 &   3.54 \\
    $x_H$     & 2.356   & 0.532 &    1.145 &    5.615 \\
    $z_H$     & -1.298  & -0.781 &   -1.340 &   -1.776 \\
    $\alpha_H$ & 0.572  & -1.265 & 1.571  & 0.238 \\
    $k_{Haa}$ & 0.186  & 0.0491 &   0.000* &       2.846 \\
    $k_{Hbb}$ & 1.259   & 1.145 &     2.006 &    0.168 \\
    $k_{Hyy}$ & 0.636  & 0.383 &    0.818 &   0.669 \\
    \midrule
    $m_F$     & 1.62  & 4.40  & 2.64  & 5.51  \\
    $r_F$     & 0.710 & 0.252 & 0.127* & 2.063 \\
    $k_{Faa}$ & 0.355 & 0.126 & 0.064 & 1.031 \\
    $k_{Fyy}$ & 0.710 & 0.252 & 0.127 & 2.063 \\
    \midrule
    $x_P$      & 0.765  & 0.276  &  0.526 & 1.079 \\
    $z_P$      & -3.194 & -0.453 & -0.586 &-2.662 \\
    $\alpha_P$ & -0.661 & 1.150  & -0.836 & 1.558 \\
    \midrule
    $m_R$   & 12.63  & 8.36  &  6.31 & 2.25 \\
    $r_R$   & 0.958  & 0.506 & 0.146 & 3.027 \\
    $k_{Raa}$ & 0.479  & 0.253 & 0.073 & 1.514 \\
    $k_{Ryy}$ & 0.958  & 0.506 & 0.146 & 3.027 \\
    \bottomrule
  \end{tabular}
\end{table}
%
\begin{figure}
  \centering
  \includegraphics[width=6in]{../figures/optimal-geometries.png}
  \label{fig:optimal-geometries}
  \caption{Depictions of the bicycle geometry and geometric representations of
    the inertial quantities for the reference bicycle and four optimal solutions
    at 3, 5, 7, and 9 m/s. Five rigid bodies are shown for each bicycle: front
    wheel (orange), rear wheel (purple), rear frame (blue), front frame
    (green), and person (red). The solid black lines represent the essential
    bicycle geometry. The dotted black line represents the steer axis. The
    solid colored curves represent the contours of solid ellipsoids with
    equivalent inertia as the principal inertia of the associated rigid body.
    The dotted colored lines represent the extents of the centroidal radii of
    gyration of each rigid body.}
\end{figure}

The bicycle optimized for 3~\si{\meter\per\second} is about three times larger
than the reference bicycle. The wheelbase is about 3~\si{\meter} and it has a
relatively large postive trail (0.7~\si{\meter}). The person is rearward on the
bike and over 3~\si{\meter} above the ground. The mass of the front frame is
the minimum posible value and the mass of the rear frame is also small.

The bicycle optimized for 5~\si{\meter\per\second} has a similar geometric
scale as the reference bicycle except that the front wheel is smaller and the
rear wheel is larger. The person is located low to the ground and tipped back,
like one might see on a recumbent bicycle.  This significantly increases the
vehicles yaw moment of inertia. The trail is miminal, but suprisingly slight
negative (5~\si{\milli\meter}). The steer axis tilt is slightly shallower than
the reference bicycle. The mass of the rear frame is similar to the reference
bicycle but the front frame is at the bound and much lower. The minor principal
axis of the front frame is almost aligned with the steer axis.

The bicycle optimized for 5~\si{\meter\per\second} is one of the three that is
geometrically smaller than the reference bicycle. The wheelbase is slightly
larger but the wheels are approximately half the diameter. The rider is low to
the ground and tipped forward. There is effiectively no trail, the steer axis
is vertical, and the (very) minimal principal axis is aligned with the steer
axis making the steering inertia-less. The roll moment of inertia is larger.

The bicycle optimized for 5~\si{\meter\per\second} is much larger than the
reference bicycle: wheelbase and rear wheel diameter of 6~\si{\meter}, front
wheel diameter of 4~\si{\meter}. The trail is large and negative, surely making
the bicycle open-loop unstable.

The parameters of the bicycles can be distilled into open loop eigenvalues for
a clearer understanding of the dynamics, shown in
Figure~\ref{fig:optimal-eigenvalues}. The 3 and 9 design speed have a simlar
eigenvalue pattern as well as the 5 and 7. The 7~\si{\meter\per\second} design
is self-stable at speeds greater than 1.2~\si{\meter\per\second} but the other
three designs are not self-stable at any travel speed from
\SIrange{0}{10}{\meter\per\second}. Both the 5 and 7 designs have a weave mode
that increases in frequency rapidly with speed. The weave frequency of the 3
m/s design is higher than the reference design and present at all speeds.
%
\begin{figure}
  \centering
  \includegraphics[width=6in]{../figures/optimal-eigenvalues.png}
  \label{fig:optimal-eigenvalues}
  \caption{Real (solid) and imaginary (dashed) parts of the open-loop eigenvlues plotted versus
    travel speed. The vertical dashed black line indicates the design speed.}
\end{figure}

\section{Discussion and Conclusion}
%
We have demonstrated the ability to find optimal parameter values of the linear
Whipple-Carvallo bicycle model under constraints that enforce a physically
realizable bicycle with an objective of improving lateral handling qualities.
We showcase four optimal bicycle designs for a range of target travel speeds.
The resulting bicycles are similar to the familiar and popular bicycle design
but have some oddities: very large size relative to the person, large negative
trail, large positive trail, minimal steer inertia, recumbent rider
orientation, and both very low and high rider mass center locations. Three of
the four bicycles are not open-loop self stable for any speed up to
10~\si{\meter\per\second} and the weave model frequencies are significantly
larger than the reference bicycle but all bicycles are controllable and have a
minimal HQM as defined in~\cite{Hess2012}.

The bicycles are reasonably physically realizable but some additional
constraints could be introduced to further improve this. In the
5~\si{\meter\per\second} design the person's body overlaps the wheels, which
would be almost impossible to realize. A constraint that ensures the torso
cannot occupy the same space as the wheels would solve this. The masses of the
front and rear frames are low with respect to the geometrical spread a steel
space frame would have to occupy. The frames also need to be able to span the
space between the respecitve wheel, mass center of the frame, and the steer
axis and for the rear frame span to the rider's support locations. We currently
only expliclty deal with spanning the space between the mass center of the
frame and the respective wheel. Another more restricting appraoch would be to
specify a generic geometric structure for the front and rear frames. If done,
maximal stress and deflections of the frames under load could also be
minimized.

Each optimizal solution requires approximately 12 hours of computation time on
a high end consumer desktop machine running on a single thread. The number of
iterations are typically in the hundreds of thousands for convergence. There
are many avenues for speeding this up that could likely result in solutions in
one or two hours on the same machine but this is still slow to be particualrly
pleasant to utlizie in the process of designing a bicycle. Furthermore, the
time required to translate the resulting parameters in to an actual bicycle
design complete with structural destails is extremely time consuming.
Never-the-less, the method shows promise for optimizing an entire vehicle for
optimal dynamics. This method can be applied to a whole host of human operated
vehicles opening up many new designs, but as with any optimization it only
captures a very small set of the variables that a designer has to take into
account for a vehicle.

\section{Reproducibility}
%
All of the source code, data, and documents needed to reproduce the presented
results and this paper can be found at the repository hosted at
\url{https://github.com/moorepants/BMD2019}.

\bibliographystyle{acm}
\bibliography{BMD2019}

\end{document}
