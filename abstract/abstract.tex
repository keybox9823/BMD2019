\documentclass{bmd2019a}

\begin{document}

\begin{flushleft}
{\fontsize{16pt}{20pt}\selectfont%
  Expanded Optimization for Discovering\\}
{\fontsize{16pt}{20pt}\selectfont%
  Optimal Lateral Handling Bicycles}
\end{flushleft}

%%%%%%%%%%%%%%%% authors %%%%%%%%%%%%%%%
\begin{flushleft}
  {\fontsize{12}{14}{Jason K. Moore, Mont Hubbard, and Ronald A. Hess}\\}
  \textit{Mechanical and Aerospace Engineering\\
          University of California, Davis\\
          One Shields Avenue, Davis, CA, USA 95817\\
          e-mail: jkm@ucdavis.edu, mhubbard@ucdavis.edu, rahess@ucdavis.edu}
\end{flushleft}

\section*{Abstract}
%
Physical design features of ground vehicles can affect their lateral handling
qualities. Geometry, mass, and mass distribution of the vehicle's primary
components as well as tire characteristics are primary contributors to poor and
good handling due to their important influence on the vehicle's dynamics. In
past work, we have presented a theoretical and computational framework for
assessing the lateral task-independent handling qualities of simplified single
track vehicle designs~\cite{Hess2012,Moore2012}. In subsequent work, we showed
that minimizing our handling quality metric (HQM) can produce theoretically
optimal handling designs when only four geometric parameters are explored as
the optimization variables~\cite{Moore2016}. The present work's goal is to
expand this optimization problem to 24 geometry, mass, and inertial parameters
as the optimization variables so that a broader search space of optimal bicycle
designs is considered. This task is complicated by the fact that we desire to
limit the search space to physically realizable designs. Doing so introduces 35
linear and non-linear constraint equations on the optimization variables. We
formulate a constrained optimization problem and use nonlinear programming to
discover optimal, yet realizable, bicycle designs.

Our formulation exposes 24 optimization variables
$\mathbf{p}\in\mathbb{R}^{24}$ that are both reformulations of and additions to
the parameters of the benchmark bicycle~\cite{Meijaard2007}. We separate the
rider's inertial parameters from the rear frame of the bicycle for the purposes
of rider/bicycle relative configuration but do not introduce any more than the
benchmark's four rigid bodies. The variables are as follows;
%
\begin{itemize}
  \itemsep-0.25em
  \item wheelbase, trail, steer axis tilt, wheel radii
  \item somersault angle of the person (same direction as bicycle pitch)
  \item masses of the front frame, rear frame, front wheel, and rear wheel
  \item mass center locations of the front frame, rear frame, and rider
  \item principal central radii of gyration of front and rear frames
  \item principal inertia axis pitch angles of the front frame, rear frame, and rider
\end{itemize}

We formulate bounds on the variables as well as linear and nonlinear
constraints among the variables to ensure a physically realizable bicycle.
Below the constraints are defined and grouped by the associated rigid body or
collection thereof:
%
\begin{description}

  \item[Vehicle] The resulting combination of the five rigid bodies.
    \begin{itemize}
      \itemsep0em
      \item The physical extents of the bicycle and rider must exist above the
        ground plane.
      \item Both bicycle and rider are symmetric about the sagittal
        plane.
      \item The mass of each bicycle rigid body is positive, greater than a
        minimum value, and the total mass is below a reasonably lift-able
        amount.
      \item The wheels cannot overlap, i.e. wheelbase > front radius + rear
        radius.
      \item The bicycle cannot topple forward during hard braking or backward
        during hard acceleration, i.e. -1 g < acceleration < 1 g.
    \end{itemize}

  \item[Wheels] Both front and rear wheels have identical constraints.
    \begin{itemize}
      \itemsep0em
      \item Wheel radius must be greater than a minimum value.
      \item Wheels are inertially wheel-like, i.e.  $I_{rot}=mr^2,
        I_{rad}=I_{rot}/2$.
    \end{itemize}

  \item[Frames] Both front and rear frames have identical constraints.
    \begin{itemize}
      \itemsep0em
      \item A pair of principal directions lie in the roll/yaw
        plane to maintain symmetry and one of these axes is within $\pm \pi/4$
        rad of the ground plane.
      \item The mass and inertia of the frames are positive and large enough to
        be constructed from materials in a space frame of specified minimal
        density.
      \item The frames are planar in nature, i.e.
        $k_{lateral}^2=k_{longitudinal}^2+k_{vertical}^2$, where $k$ is the
        radius of gyration.
    \end{itemize}

  \item[Rider] A single rigid body represents the rider.
    \begin{itemize}
      \itemsep0em
      \item Rider mass is that of an average person.
      \item The rider's joint angles are fixed in a nominal configuration
        typical of upright bicycling and the resulting mass distribution is
        derived from standard body segment estimation methods.
      \item One rider principal axis is parallel with the bicycle pitch axis to
        enforce symmetry.
    \end{itemize}

\end{description}

The constraint statements above translate into upper and lower bounds on 12 of
the 24 variables $\mathbf{p}^U,\mathbf{p}^L$ and 14 linear and 21 nonlinear
constraint $\mathbf{g}(\mathbf{p})$ among the variables and thus can be
expressed in the standard non-linear programming form.
%
\begin{equation}
  \begin{aligned}
    & \underset{\mathbf{p}}{\text{minimize}} & & max(\textrm{HQM}(\mathbf{p})) \\
    & \text{subject to} & & \\
    & & & \mathbf{p}^L \leq \mathbf{p} \leq \mathbf{p}^U \\
    & & & \mathbf{g}^L \leq \mathbf{g}(\mathbf{p}) \leq \mathbf{g}^U \\
  \end{aligned}
\end{equation}

We make use of the interior-point optimizer IPOPT~\cite{Wachter2006} to find
solutions to this nonlinear programming problem at a variety of vehicle speeds
and will present the resulting discovered optimal bicycle designs in the full
conference paper.

\bibliographystyle{acm}
\bibliography{BMD2019}

\end{document}
